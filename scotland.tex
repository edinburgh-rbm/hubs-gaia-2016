%\narrative{In this section we describe CIX as implemented by HUBS, a non-profit etc, etc... Then (1) Particulars of implementation in Scotland and how they were dealth with (if they exist). (2) What the current network looks like, and anticipated changes and/or {\bf unanticipated stuffs, lessons learned}. (3) Outcomes as they exist so far.}

In this section we describe our first implementation of RemIX in a series of planned deployments across Scotland. In the West Highlands there
is a cluster of
%%
%% Mull
%% Moidart
%% Small Isles
%% Knoydart
%% Sleat
%% Tegola (experimental, lab)
%% Tegola (production)
%% Glenelg
%% Applecross
%% CMNet
%% Lochiel
%%
11 small community networks. Their spread across \textasciitilde 2000km$^2$
%% source -- 100km long by 20km wide
of sea and mountainous islands makes the
construction of an exchange fabric geographically ambitious. Four networks have
a history of interconnecting and sharing network resources, pre-established
relationships that must be respected in our deployment.

Our deployment's location is its namesake, the West Highlands Internet
Exchange (WHIX). Both logical and physical layers are described below, with
additional lessons and comments drawn from our experience.

\subsection{West Highland IX at Layer 1}

The physical \ac{WHIX} fabric is overlayed onto a map of the region in
Figure~\ref{fig:whixmap}. The map itself is stylized for clarity, and
preserves critical geographical features. Red connected nodes indicate
the connection sites. In a traditional IXP these sites would be the
ethernet ports into which subscriber ASes plug-in. WHIX sites are
connected by wirless radio links in black, and leased 100Mbps or 1Gbps
circuits in orange. The areas enclosed with dotted lines correspond to
the service areas from each member network.
%%%%% FIGURE %%%%%
Figure~\ref{fig:whixmap}.
\begin{figure*}
  \centering
  \subfloat[Physical topology of \ac{WHIX}.]{
    \resizebox{0.9\columnwidth}{0.35\textheight}{
      \begin{tikzpicture}
        \whixphysicaldiagram
      \end{tikzpicture}
    }
    \label{fig:whixmap}
  }\hspace{\columnsep}
  \subfloat[Subscriber connections to \ac{WHIX}]{
    \resizebox{0.9\columnwidth}{0.35\textheight}{
      \begin{tikzpicture}
        \whixmeshdiagram
      \end{tikzpicture}
     }
    \label{fig:phytop}
  }
  \caption{Physical and logical layout of \ac{WHIX}. In
  Figure~\ref{fig:whixmap} the dark lines correspond to radio links
  and the light, curved lines to leased ethernet circuits.
  In Figure~\ref{fig:phytop} the dashed lines
  correspond to internal layer-2 circuits forming \ac{WHIX}
  switching fabric and the solid lines to member connections.}
\end{figure*}

Where Figure~\ref{fig:whixmap} representats the physical topology
of \ac{WHIX} itself, Figure~\ref{fig:phytop} shows the member network
connections. In the latter, unlabeled red nodes are the \ac{WHIX}
points of presence and the dashed lines represent the fully connected
virtual topology that implements the exchange E-LAN.

The two places in the region where long-distance ethernet circuits are
available on the mainland are the towns of Mallaig and Kyle of
Lochalsh. Circuits\footnote{At the time of writing, the circuit from
Mallaig is in place, and that from Kyle is planned.} from these sites
connect back to the Pulsant datacentre in Edinburgh to facilitate
remote peering --- and indeed the provision of Internet access via the
exchange point.

The radio links are implemented with equipment from Ubiquiti Networks,
configured in transparent bridge mode so that they can be considered
simply as Ethernet from a functional perspective. The switching fabric
itself at each of \ac{WHIX} points of presence is implemented with
Mikrotik routers. This choice was made on the grounds of their
moderate port density, low power consumption, low cost, and adequately
featureful \ac{MPLS} implementation, of which more below.

\subsection{West Highland IX at Layers 2/3}

To operate the exchange, various network numbers are required. It is
not appropriate to use private IP addresses~\cite{rfc1918}
or \acp{ASN}~\cite{rfc6996} for these purposes because of the
possibility --- likelihood even --- of collision. A member network is
perfectly entitled to use whichever numbers they like from the
reserved rage internally for their own use, and if it happens that
those same numbers are in use on the exchange, or even transitively
reachable via other exchange members, it creates an ambiguous logical
topology. This can be especially severe in the case of IP network
collisions where the duplicate network is learned via external eBGP:
the path selection algorithm will prefer the foreign path to the local
one!\footnote{The authors have long wondered about the wisdom of these
defaults in the path selection algorithm and have never encountered a
situation where the default behaviour is the desired
one. Nevertheless, this is what we have.}

Requesting IP address allocations for exchange use from the RIPE NCC
was straightforward. It did, however, require the formal development
of rules for connecting to the exchange~\cite{whixrules}.


%%%%% FIGURE %%%%%
\begin{figure}[h]
  \resizebox{\linewidth}{!}{
    \begin{tikzpicture}
      \whixtopodiagram
    \end{tikzpicture}
  }
  \caption{
  \textbf{Autonomous System topology.} The members of a RemIX form a fully connected network where each may communicate
  with another over the exchange without intermediation. Unusually for \acp{IXP} it is common practice to provide Internet transit over the exchange for the vast majority of members that require this. Note as
  well the private interconnection, outwith the exchange, between Skyenet
  and Hebnet. Such private interconnections are typically made as
  an optimisation where it is not feasible to do so efficiently over the exchange.
  }
\end{figure}



\subsection{Usability 'Trade-off'} \label{subsec:use}
\narrative{Our use of cheap proprietary vs open batman-compatible}


\subsection{Deployment Discussion}

Our experience motivates higher-level comments to further distinguish RemIX
deployments form their larger \ac{IXP} and \ac{NAP} cousins. Flat networks
consisting of a single layer-2 broadcast domain can be plagued by difficult to
troubleshoot problems. By its very design RemIX requires that members be able to
communicate directly without mediation at the IP layer. This similarity is
superficial. RemIX eliminates a large class of potential problems by allowing
only unicast and \ac{ARP} traffic on the exchange. Moreover, members must
nominate a specific \ac{MAC} address for their connections, which reduces the
risk of loops and broadcast storms. We also adopt \ac{IXP} best practices such
as quarantines for new connections while they are evaluated for correctness.

% Many working network engineers have had the experience of inheriting a flat
% network consisting of a single layer-2 broadcast domain, invariably plagued by
% difficult to troubleshoot problems.

% The architecture is such that the extension of the layer-2 broadcast domain
% extends far enough to meet the design requirements but not so far as to become
% unwieldy.

The RemIX norm of providing IP transit also deserves to be addressed. Transit
via the exchange, for networks that are not otherwise visible on the Internet,
may evoke notions of conflicting interests that beset \acp{NAP}.  With regards
to our Scottish deployments the similarity is superficial. Our transit provider,
HUBS CiC, is a cooperative that exists for the benefit of its members. In our
deployment, HUBS provides service only to member networks and never to
end-users. As a consequence all parties' economic interests are aligned.

%\narrative{In this section we describe CIX as implemented by HUBS, a non-profit etc, etc... Then (1) Particulars of implementation in Scotland and how they were dealth with (if they exist). (2) What the current network looks like, and anticipated changes and/or {\bf unanticipated stuffs, lessons learned}. (3) Outcomes as they exist so far.}

We have implemented the RemIX model in the West Highlands where there
is a cluster of
%%
%% Mull
%% Moidart
%% Small Isles
%% Knoydart
%% Sleat
%% Tegola (experimental, lab)
%% Tegola (production)
%% Glenelg
%% Applecross
%% CMNet
%% Lochiel
%%
11 small community networks. They are spread over an area of some 400km$^2$
\narrative{Is this number right?} of sea and mountainous islands, making the
construction of an exchange fabric geographically ambitious. Four networks have
a history of interconnecting and sharing network resources, pre-established
relationships that must be respected in our deployment.

The location of our deployment is its namesake, the West Highlands Internet
Exchange (WHIX). Both logical and physical layers are described, with
additional lessons and comments drawn from our experience.



\subsection{West Highlands IX at Layer 1}

The full WHIX topology is best represented from the two perspectives, depicted
across Figures~\ref{fig:whixmap} and~\ref{fig:phytop}.

The physical WHIX fabric is overlayed onto a map of the region in
Figure~\ref{fig:whixmap}.. The map itself is stylized for clarity, and preserves
critical geographical features. Red connected nodes indicate the connection
sites. In a traditional IXP these sites would be the ethernet ports into which
subscriber ASes plug-in. WHIX sites are connected by wirless radio links in
black, and leased 100Mbps or 1Gbps circuits in orange. The areas enclosed with
dotted lines correspond to the visible service areas from each site.
\narrative{Does this note matter?:} Note in particular how these service areas
correspond more  closely to maritime features than landforms.

%%%%% FIGURE %%%%%
Figure~\ref{fig:whixmap}.
\begin{figure*}
  \centering
  \subfloat[Physical topology of \ac{WHIX}.]{
    \resizebox{0.9\columnwidth}{0.35\textheight}{
      \begin{tikzpicture}
        \whixphysicaldiagram
      \end{tikzpicture}
    }
    \label{fig:whixmap}
  }\hspace{\columnsep}
  \subfloat[Subscriber connections to \ac{WHIX}]{
    \resizebox{0.9\columnwidth}{0.35\textheight}{
      \begin{tikzpicture}
        \whixmeshdiagram
      \end{tikzpicture}
     }
    \label{fig:phytop}
  }
  \caption{Physical and logical layout of \ac{WHIX}. In
  Figure~\ref{fig:whixmap} the dark lines correspond to radio links
  and the light, curved lines to leased ethernet circuits.
  In Figure~\ref{fig:phytop} the dashed lines
  correspond to internal layer-2 circuits forming \ac{WHIX}
  switching fabric and the solid lines to member connections.}
\end{figure*}

Figure~\ref{fig:whixmap} is an accurate representation of the WHIX fabric as it
is distributed across the region. To be clear, Figure~\ref{fig:whixmap} \emph{is
the ethernet switch}. The physical relationship between WHIX and its subscriber
access networks is shown in Figure~\ref{fig:phytop}.
%necessarily more complex. It is designed to circumvent geographical barriers,
%while respecting land ownership and local by-laws.
In the Figure~\ref{fig:phytop} view of the topology, unlabeled red nodes are the
connection sites that correspond directly with the red node connection sites in
Figure~\ref{fig:whixmap}. The dashed lines in Figure~\ref{fig:phytop} are the
layer-2 WHIX circuits that form a fully connected mesh linking connection sites.
The solid nodes are the remote access networks labeled with their names. They
establish and are responsible for links to WHIX sites, represented by the solid
lines.

The two places in the region where long-distance ethernet circuits are
available on the mainland are the towns of Mallaig and Kyle of
Lochalsh. Circuits\footnote{At the time of writing, the circuit from
Mallaig is in place, and that from Kyle is planned.} from these sites
connect back to the Pulsant datacentre in Edinburgh to facilitate
remote peering --- and indeed the provision of Internet access via the
exchange point.


From these sites, a series of wireless ethernet circuits connects
seven other \ac{WHIX} points of presence as shown in



At each of these points two or more members may connect to the
exchange. Indeed when deciding whether a particular site should be
a \ac{WHIX} point of presence the criterion that was settled upon was
that at least two members must be present there, otherwise a member
connection would be made remotely.\narrative{clarify}




\subsection{West Highlands IX at Layers 2/3}

To operate the exchange, various network numbers are required. It is
not appropriate to use private IP addresses~\cite{rfc1918}
or \acp{ASN}~\cite{rfc6996} for these purposes because of the
possibility --- likelihood even --- of collision. A member network is
perfectly entitled to use whichever numbers they like from the
reserved rage internally for their own use, and if it happens that
those same numbers are in use on the exchange, or even transitively
reachable via other exchange members, it creates an ambiguous logical
topology. This can be especially severe in the case of IP network
collisions where the duplicate network is learned via external eBGP:
the path selection algorithm will prefer the foreign path to the local
one!\footnote{The authors have long wondered about the wisdom of these
defaults in the path selection algorithm and have never encountered a
situation where the default behaviour is the desired
one. Nevertheless, this is what we have.}

Requesting IP address allocations for exchange use from the RIPE NCC
was straightforward. It did, however, require the formal development
of rules for connecting to the exchange~\cite{whixrules}.


%%%%% FIGURE %%%%%
\begin{figure}[h]
  \resizebox{\linewidth}{!}{
    \begin{tikzpicture}
      \whixtopodiagram
    \end{tikzpicture}
  }
  \caption{
  \textbf{Autonomous System topology.} The members of a RemIX form a fully connected network where each may communicate
  with another over the exchange without intermediation. Unusually for \acp{IXP} it is common practice to provide Internet transit over the exchange for the vast majority of members that require this. Note as
  well the private interconnection, outwith the exchange, between Skyenet
  and Hebnet. Such private interconnections are typically made as
  an optimisation where it is not feasible to do so efficiently over the exchange.
  }
\end{figure}



\subsection{Usability 'Trade-off'} \label{subsec:use}
\narrative{Our use of cheap proprietary vs open batman-compatible}


\subsection{Use of SDN}
\narrative{describe our use of Ansible; also larger implications for SDN}

%\narrative{In this section we describe CIX as implemented by HUBS, a non-profit etc, etc... Then (1) Particulars of implementation in Scotland and how they were dealth with (if they exist). (2) What the current network looks like, and anticipated changes and/or {\bf unanticipated stuffs, lessons learned}. (3) Outcomes as they exist so far.}

In this section we describe our first implementation of RemIX in a series of planned deployments across Scotland. In the West Highlands there
is a cluster of
%%
%% Mull
%% Moidart
%% Small Isles
%% Knoydart
%% Sleat
%% Tegola (experimental, lab)
%% Tegola (production)
%% Glenelg
%% Applecross
%% CMNet
%% Lochiel
%%
11 small community networks. Their spread across \textasciitilde 2000km$^2$
%% source -- 100km long by 20km wide
of sea and mountainous islands makes the
construction of an exchange fabric geographically ambitious. Four networks have
a history of interconnecting and sharing network resources, pre-established
relationships that must be respected in our deployment.

Our deployment's location is its namesake, the West Highlands Internet
Exchange (WHIX). Both logical and physical layers are described below, with
additional lessons and comments drawn from our experience.

\subsection{West Highland IX at Layer 1}

The physical \ac{WHIX} fabric is overlayed onto a map of the region in
Figure~\ref{fig:whixmap}. The map itself is stylized for clarity, and
preserves critical geographical features. Red connected nodes indicate
the connection sites. In a traditional IXP these sites would be the
ethernet ports into which member ASes plug-in. WHIX sites are
connected by wirless radio links in black, and leased 100Mbps or 1Gbps
circuits in orange. The areas enclosed with dotted lines correspond to
the service areas from each member network.
%%%%% FIGURE %%%%%
\begin{figure*}
  \centering
  \subfloat[Physical topology of \ac{WHIX}.]{
    \resizebox{0.9\columnwidth}{0.35\textheight}{
      \begin{tikzpicture}
        \whixphysicaldiagram
      \end{tikzpicture}
    }
    \label{fig:whixmap}
  }\hspace{\columnsep}
  \subfloat[Member connections to \ac{WHIX}]{
    \resizebox{0.9\columnwidth}{0.35\textheight}{
      \begin{tikzpicture}
        \whixmeshdiagram
      \end{tikzpicture}
     }
    \label{fig:phytop}
  }
  \caption{Physical and logical layout of \ac{WHIX}. In
  Figure~\ref{fig:whixmap} the dark lines correspond to radio links
  and the light, curved lines to leased ethernet circuits.
  In Figure~\ref{fig:phytop} the dashed lines
  correspond to internal layer-2 circuits forming \ac{WHIX}
  switching fabric and the solid lines to member connections.}
\end{figure*}

Where Figure~\ref{fig:whixmap} represents the physical topology
of \ac{WHIX} itself, Figure~\ref{fig:phytop} shows the member network
connections. In the latter, unlabeled red nodes are the \ac{WHIX}
points of presence and the dashed lines represent the fully connected
virtual topology that implements the exchange E-LAN.

The two places in the region where long-distance ethernet circuits are
available on the mainland are the towns of Mallaig and Kyle of
Lochalsh. Circuits\footnote{At the time of writing, the circuit from
Mallaig is in place, and that from Kyle is planned.} from these sites
connect back to the Pulsant datacentre in Edinburgh to facilitate
remote peering --- and indeed the provision of Internet access via the
exchange point.

The radio links are implemented with equipment from Ubiquiti Networks,
configured in transparent bridge mode so that they can be considered
simply as Ethernet from a functional perspective. The switching fabric
itself at each of \ac{WHIX} points of presence is implemented with
Mikrotik routers. This choice was made on the grounds of their
moderate port density, low power consumption, low cost, and adequately
featureful \ac{MPLS} implementation, of which more below. All
equipment is configured to pass Ethernet frames of at least 1600 bytes
to provide room for the necessary extra protocol headers for
implementing the E-LAN service.

\subsection{West Highland IX at Layer 2}
\label{sec:bgpvpls}

The design requirement of functional equivalence to a \ac{MAC} address
learning Ethernet switch is met using \ac{BGP}
signalled \ac{VPLS}~\cite{rfc4761} to create a full set of \ac{LSP}
pseudo-wires between every pair of \ac{WHIX} edge routers. It should
be emphasised that this section is implementation detail internal
to \ac{WHIX} and is not visible to the members: all they see is a
big Ethernet switch. Neither is this the only way to meet the design
requirement. 

Each \ac{WHIX} router maintains an \ac{OSPF} routing protocol
adjacency with its neighbours and distributes reachability information
for its loopback IP address. All addresses used for this purpose are
private IPv4 addresses~\cite{rfc1918}. This is the basic layer that
ensures reachability throughout the distributed fabric. With IP
connectivity established, \ac{LDP}~\cite{rfc5036} is used to enable
the carriage of non-IP traffic with \ac{MPLS} labels according to the
topology of the underlying \ac{OSPF} network.

Each router then establishes a \ac{BGP} peering session with the
routers at Mallaig and Sabhal M\`{o}r Ostaig which act as route
reflectors~\cite{rfc4456}. The use of route-reflectors enables all
participating routers to exchange reachability information without
requiring a full mesh ($n^2$) of internal peering
sesions. With \ac{BGP} signalling established throughout the \ac{WHIX}
fabric,  the multi-protocol extensions~\cite{rfc4760} can then be used
by each router to signal that it wishes to form part of the
exchange \ac{LAN}, which has a defined route target value and in this
way a fully meshed \ac{VPLS} is created.

Now each router has a virtual bridge interface that forms part of the
exchange \ac{LAN}, and physical interfaces to which members are
connected can be added to this bridge so that they form part of the
exchange. Care must be taken to prevent loops so that members do not
see traffic that they themselves originated and this is accomplished
with a split-horizon method~\cite{rfc4762}. Equally care must be taken
that members do not themselves create bridge loops via their own
network so a \ac{MAC} address filter on the port is employed.

\subsection{West Highland IX at Layer 3}

Logical connectivity being established between all member ports, it is
necessary to assign IP addresses to their border routers and public
exchange infrastructure such as the route server. As mentioned above
the use of private IP address space for this purpose is undesireable
due to the possibility of collisions with members' own
infrastructure. Fortunately by design, \ac{WHIX} meets the definition
of an \ac{IXP}~\cite{ripe451,whixrules} and allocations of IPv4 and
IPv6 addresses were obtained from the RIPE NCC~\cite{ripe649}.

Each member network now has everything they need: the ability to
communicate at layer 2 with each other, an IP address for layer 3, an
autonomous system number to identify themselves, and their own
networks to announce.
%%%%% FIGURE %%%%%
\begin{figure}[h]
  \resizebox{\linewidth}{!}{
    \begin{tikzpicture}
      \whixtopodiagram
    \end{tikzpicture}
  }
  \caption{
  Autonomous System topology. The members of a RemIX form a fully
  connected network where each may communicate with another over the
  exchange without intermediation.
  %% Unusually for \acp{IXP} it is
  %% common practice to provide Internet transit over the exchange for
  %% the vast majority of members that require this. Note as well the
  %% private interconnection, outwith the exchange, between Skyenet and
  %% Hebnet. Such private interconnections are typically made as an
  %% optimisation where it is not feasible to do so efficiently over the
  %% exchange.
  }
\end{figure}
So that they don't need to explicitly configure bilateral peering
relationships among each other (another $n^2$ configuration
task), \ac{WHIX} provides two route-servers, again at Mallaig and
Sabhal M\`{o}r Ostaig to redistribute reachability information. A
route-server behaves like a route-reflector except that it does not
insert its own \ac{ASN} into the path.

HUBS is also present at \ac{WHIX} as a member, but in addition to
public multilateral peering, it establishes bilateral sessions with
those members that wish where it will announce either a default route
or full Internet routing tables and will forward member's
announcements to its other peers and upstreams. In this way transit
and so connectivity to the global Internet is provided over the
exchange.
%\subsection{Usability 'Trade-off'} \label{subsec:use}
%\narrative{Our use of cheap proprietary vs open batman-compatible}


\subsection{Deployment Discussion}

Our experience motivates higher-level comments to further distinguish
RemIX deployments form their larger \ac{IXP} cousins. Flat networks
consisting of a single layer-2 broadcast domain can be plagued by
difficult to troubleshoot problems. By its very design RemIX requires
that members be able to communicate directly without mediation at the
IP layer. Like other \acp{IXP} RemIX eliminates a large class of
potential problems by allowing only unicast and \acs{ARP} traffic on
the exchange. Moreover, members must nominate a specific \acs{MAC}
address for their connections, which reduces the risk of loops and
broadcast storms. We also adopt best practices such as quarantines for
new connections while they are evaluated for correctness.

% Many working network engineers have had the experience of inheriting a flat
% network consisting of a single layer-2 broadcast domain, invariably plagued by
% difficult to troubleshoot problems.

% The architecture is such that the extension of the layer-2 broadcast domain
% extends far enough to meet the design requirements but not so far as to become
% unwieldy.

The RemIX norm of providing IP transit also deserves to be addressed. Transit
via the exchange, for networks that are not otherwise visible on the Internet,
may evoke notions of conflicting interests that beset \acp{NAP}.  With regards
to our Scottish deployments the similarity is superficial. Our transit provider,
HUBS CiC, is a cooperative that exists for the benefit of its members. In our
deployment, HUBS provides service only to member networks and never to
end-users. As a consequence all parties' economic interests are aligned.

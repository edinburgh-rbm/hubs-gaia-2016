%\narrative{In this section we describe CIX as implemented by HUBS, a non-profit etc, etc... Then (1) Particulars of implementation in Scotland and how they were dealth with (if they exist). (2) What the current network looks like, and anticipated changes and/or {\bf unanticipated stuffs, lessons learned}. (3) Outcomes as they exist so far.}

We have implemented the RemIX model in the West Highlands where there
is a cluster of
%%
%% Mull
%% Moidart
%% Small Isles
%% Knoydart
%% Sleat
%% Tegola (experimental, lab)
%% Tegola (production)
%% Glenelg
%% Applecross
%% CMNet
%% Lochiel
%%
11 small community networks, four of which have a history of
interconnecting and sharing network resources. They are spread over an
area of some 400km$^2$ \narrative{Is this number right?} making the
construction of an exchange fabric geographically ambitious. It is
called the \acf{WHIX}.

This project received funding from the Scottish Government, and a
special-purpose not for profit organisation \ac{WHAN} was created to
operate the exchange. During its incubation period, it is jointly
operated by HUBS and the University of Edinburgh.

\subsection{RemIX in Scotland}

The two places in the region where long-distance ethernet circuits are
available on the mainland are the towns of Mallaig and Kyle of
Lochalsh. Circuits\footnote{At the time of writing, the circuit from
Mallaig is in place, and that from Kyle is planned.} from these sites
connect back to the Pulsant datacentre in Edinburgh to facilitate
remote peering --- and indeed the provision of Internet access via the
exchange point.

From these sites, a series of wireless ethernet circuits connects
seven other \ac{WHIX} points of presence as shown in
Figure~\ref{fig:whixmap}. 
\begin{figure}[h]
  \resizebox{\linewidth}{!}{
    \begin{tikzpicture}
      \whixphysicaldiagram
    \end{tikzpicture}
  }
  \caption{
  Diagram of the \acf{WHIX}. Red markers indicate
  the sites at which networks may connect, black lines indicate radio
  links, and orange lines leased 100Mbps or 1Gbps circuits. The areas
  enclosed with dotted lines correspond to the service areas of member
  networks. Note in particular how these service areas correspond more 
  closely to maritime features than landforms.
  }
  \label{fig:whixmap}
\end{figure}
At each of these points two or more members may connect to the
exchange. Indeed when deciding whether a particular site should be
a \ac{WHIX} point of presence the criterion that was settled upon was
that at least two members must be present there, otherwise a member
connection would be made remotely.\narrative{clarify}

To operate the exchange, various network numbers are required. It is
not appropriate to use private IP addresses~\cite{rfc1918}
or \acp{ASN}~\cite{rfc6996} for these purposes because of the
possibility --- likelihood even --- of collision. A member network is
perfectly entitled to use whichever numbers they like from the
reserved rage internally for their own use, and if it happens that
those same numbers are in use on the exchange, or even transitively
reachable via other exchange members, it creates an ambiguous logical
topology. This can be especially severe in the case of IP network
collisions where the duplicate network is learned via external eBGP:
the path selection algorithm will prefer the foreign path to the local
one!\footnote{The authors have long wondered about the wisdom of these
defaults in the path selection algorithm and have never encountered a
situation where the default behaviour is the desired
one. Nevertheless, this is what we have.}

Requesting IP address allocations for exchange use from the RIPE NCC
was straightforward. It did, however, require the formal development
of rules for connecting to the exchange~\cite{whixrules}. 



\subsection{Usability 'Trade-off'} \label{subsec:use}
\narrative{Our use of cheap proprietary vs open batman-compatible}


\subsection{Use of SDN}
\narrative{describe our use of Ansible; also larger implications for SDN}



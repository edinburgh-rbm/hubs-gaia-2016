%\narrative{In this section we describe CIX as implemented by HUBS, a non-profit etc, etc... Then (1) Particulars of implementation in Scotland and how they were dealth with (if they exist). (2) What the current network looks like, and anticipated changes and/or {\bf unanticipated stuffs, lessons learned}. (3) Outcomes as they exist so far.}

In this section we describe our first implementation of RemIX in a series of planned deployments across Scotland. In the West Highlands there
is a cluster of
%%
%% Mull
%% Moidart
%% Small Isles
%% Knoydart
%% Sleat
%% Tegola (experimental, lab)
%% Tegola (production)
%% Glenelg
%% Applecross
%% CMNet
%% Lochiel
%%
11 small community networks. Their spread across \textasciitilde 2000km$^2$
%% source -- 100km long by 20km wide
of sea and mountainous islands makes the
construction of an exchange fabric geographically ambitious. Four networks have
a history of interconnecting and sharing network resources, pre-established
relationships that must be respected in our deployment.

Our deployment's location is its namesake, the \ac{WHIX}. Both logical
and physical layers are described below, with additional lessons and
comments drawn from our experience.

\subsection{West Highland IX at Layer 1}

The physical \ac{WHIX} fabric is overlayed onto a stylized map of the region in
Figure~\ref{fig:whixmap}. The map itself preserves critical geographical
features. Red connected nodes are the connection sites. In a traditional
\ac{IXP} these sites are the Ethernet ports into which subscriber \acp{AS}
plug-in. WHIX sites are connected by wireless radio links in black, and leased
100Mbps or 1Gbps circuits in orange. The areas enclosed with dotted lines
correspond to the service areas reachable from each site.
%%%%% FIGURE %%%%%
\begin{figure*}
  \centering
  \subfloat[Physical topology of \ac{WHIX}.]{
    \resizebox{0.8\columnwidth}{0.35\textheight}{
      \begin{tikzpicture}
        \whixphysicaldiagram
      \end{tikzpicture}
    }
    \label{fig:whixmap}
  }\hfil %\hspace{\columnsep}
  \subfloat[Member connections to \ac{WHIX}]{
    \resizebox{0.8\columnwidth}{0.35\textheight}{
      \begin{tikzpicture}
        \whixmeshdiagram
      \end{tikzpicture}
     }
    \label{fig:phytop}
  }
  \caption{Physical and logical layout of \ac{WHIX}. In
  Figure~\ref{fig:whixmap} the dark lines correspond to radio links
  and the light, curved lines to leased ethernet circuits.
  In Figure~\ref{fig:phytop} the dashed lines
  correspond to internal layer-2 circuits forming \ac{WHIX}
  switching fabric and the solid lines to member connections.}
\end{figure*}

We complement \ac{WHIX}' physical topology in Figure~\ref{fig:whixmap} with
the member network in Figure~\ref{fig:phytop}. In the latter, unlabeled red
nodes are the \ac{WHIX} points of presence and correspond with the same set of
red nodes in Figure~\ref{fig:whixmap}. The dashed lines represent the fully
connected virtual topology that implements the exchange E-LAN.

The two places in the region where long-distance ethernet circuits are
available on the mainland are the towns of Mallaig and Kyle of
Lochalsh. Circuits\footnote{At the time of writing, the circuit from
Mallaig is in place, and that from Kyle is planned.} from these sites
connect back to the Pulsant datacentre in Edinburgh to facilitate
remote peering --- and indeed the provision of Internet access via the
exchange point.

The radio links are implemented with equipment from Ubiquiti Networks,
configured in transparent bridge mode so that they appear as Ethernet from a
functional perspective. The switching fabric itself at each of \ac{WHIX} points
of presence is implemented with Mikrotik routers. This choice was made because
of their moderate port density, low power consumption, low cost, and adequately
featureful \acs{MPLS} implementation. We revisit this choice in the next
section. All equipment is configured to pass Ethernet frames of at least 1600
bytes to provide room for the necessary extra protocol headers for implementing
the E-LAN service.

\subsection{West Highland IX at Layer 2}\label{sec:bgpvpls}

We emphasize that layer-2 details are \emph{internal} to
\ac{WHIX}, and invisible to members who only see an Ethernet switch. Also, our
implementation decisions are by no means the only possible means of
implementation.

In \ac{WHIX} the requirement for functional equivalence to a \acs{MAC} address
learning Ethernet switch is met using \acs{BGP} signalled
\acs{VPLS}~\cite{rfc4761}. This creates a full set of \acs{LSP} pseudo-wires
between every pair of \ac{WHIX} edge routers.
% It should be emphasised that this section is implementation detail internal to
% \ac{WHIX} and is not visible to the members: all they see is a big Ethernet
% switch. Neither is this the only way to meet the design requirement.
Each \ac{WHIX} router maintains an \acs{OSPF} routing protocol
adjacency with its neighbours and distributes reachability information
for its loopback IP address. All addresses used for this purpose are
private IPv4 addresses~\cite{rfc1918}. This is the basic layer that
ensures reachability throughout the distributed fabric.
%With IP connectivity established,
Non-IP traffic is carried via \acs{LDP}~\cite{rfc5036}
%is used to enable the carriage of non-IP traffic
with \acs{MPLS} labels according to the topology of the underlying \acs{OSPF}
network.

Routers in \ac{WHIX} establish \acs{BGP} peering sessions with routers at
Mallaig and Sabhal M\`{o}r Ostaig that act as route reflectors~\cite{rfc4456}.
Participating routers use route reflectors to exchange reachability information
without requiring a full mesh ($n^2$) of internal peering sessions. The presence
of \acs{BGP} signalling throughout the \ac{WHIX} fabric enables the use of
multi-protocol extensions~\cite{rfc4760}. Routers can use extensions to signal a
desire to form part of the exchange \acs{LAN}.
%% MF- What has a defined target, the extension or the router?
%, which has a defined route target value and in this
The result is a fully meshed \acs{VPLS}, where each router has a virtual bridge
interface that forms part of the exchange \acs{LAN}.

Interfaces can be added to virtual bridges, as needed, to
form part of the exchange. Care must be taken to prevent loops in which members
see the traffic that they originate. This is accomplished with a split-horizon
method~\cite{rfc4762}. Equally, members must be prevented from creating bridge
loops via their own network by employing \acs{MAC} address filter on relevant
ports.

\subsection{West Highland IX at Layer 3+}

Given logical connectivity between all member ports, it remains to assign IP
addresses to their border routers, as well as public infrastructure such as the
router server. As mentioned above the use of private IP address space for this
purpose is undesireable since it generates risks of collisions with members' own
infrastructure. \ac{WHIX}, and more generally RemIX, is fortunate in this
regard: The design meets the definition of an \ac{IXP}~\cite{ripe451,whixrules},
making it possible to acquire IPv4 and IPv6 address allocations from RIPE
NCC~\cite{ripe649}.

%%%%% FIGURE %%%%%
\begin{figure}[h]
  \resizebox{\linewidth}{!}{
    \begin{tikzpicture}
      \whixtopodiagram
    \end{tikzpicture}
  }
  \caption{
  Autonomous System topology. The members of a RemIX form a fully
  connected network where each may communicate with another over the
  exchange without intermediation.
  %% Unusually for \acp{IXP} it is
  %% common practice to provide Internet transit over the exchange for
  %% the vast majority of members that require this. Note as well the
  %% private interconnection, outwith the exchange, between Skyenet and
  %% Hebnet. Such private interconnections are typically made as an
  %% optimisation where it is not feasible to do so efficiently over the
  %% exchange.
  }
  \label{fig:l3}
\end{figure}

The full layer-3 \ac{WHIX} topology is shown in Figure~\ref{fig:l3}. At this
stage member network have everything they need. Members can communicate at layer
2. Each has an IP address at layer 3, an autonomous system number for
identification, and their own networks to announce. Bilateral peering
arrangements (an otherwise $n^2$ configuration task) are facilitated by two
route-servers, as before at Mallaig and Sabhal M\`{o}r Ostaig. The route servers
redistribute reachability information, akin to a route-reflector omits its own
\ac{ASN} from the path.

The transit provider, HUBS (see Section~\ref{subsec:hubs}), is also
present at \ac{WHIX} as a member. In addition to public multilateral peering, it
establishes bilateral sessions with members wishing to announce either a
default route or full Internet routing tables. HUBS forwards those members'
announcements upstream and to their peers. In this way transit, and hence
connectivity to the global Internet, is provided over the exchange.
%\subsection{Usability 'Trade-off'} \label{subsec:use}
%\narrative{Our use of cheap proprietary vs open batman-compatible}


\subsection{Deployment Discussion}

Our experience motivates higher-level comments to further distinguish
RemIX deployments form their larger \ac{IXP} cousins. Flat networks
consisting of a single layer-2 broadcast domain can be plagued by
problems that are difficult to troubleshoot. By its very design RemIX requires
that members be able to communicate directly without mediation at the
IP layer. Like other \acp{IXP} RemIX eliminates a large class of
potential problems by allowing only unicast and \acs{ARP} traffic on
the exchange. Moreover, members must nominate a specific \acs{MAC}
address for their connections, which reduces the risk of loops and
broadcast storms. We also adopt best practices such as quarantines for
new connections while they are evaluated for correctness.

% Many working network engineers have had the experience of inheriting a flat
% network consisting of a single layer-2 broadcast domain, invariably plagued by
% difficult to troubleshoot problems.

% The architecture is such that the extension of the layer-2 broadcast domain
% extends far enough to meet the design requirements but not so far as to become
% unwieldy.

IP transit in RemIX also deserves to be addressed. Transit
via the exchange, for networks that are not otherwise visible on the Internet,
may evoke notions of conflicting interests that
beset \acp{NAP}. However the similarity is superficial. Here, the transit
provider and exchange operator HUBS \textsc{c.i.c.}, is a cooperative
that exists for the benefit of and is controlled by the members, who
are also members of the \ac{IXP}. As a consequence all parties'
economic interests are aligned.

Finally, \ac{WHIX}' implementation using \acs{BGP}-\acs{VPLS}
to construct the exchange fabric makes it possible to offer auxiliary
point to point pseudo-wire services to its members. This is
useful for those members that have need for making connections internal to their
networks.
% This is as simple as creating a distinct route target
% bridged to two of their ports which results in the establishment of a
% dedicated~\acs{VPLS} instance for them.

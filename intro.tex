In remote and rural regions the last-mile problem has been the subject of much
focus. Deployments in remote regions of the world have shown that it is possible
to build high quality access networks in otherwise under-serviced
regions~\cite{xxx}. Their underlying technologies range in medium (eg. copper
or fibre-optic cabling, licensed or unlicensed wireless), energy (eg. solar or
wind generation or mains supplied), and topology.
%(eg. multi-hop, one-to-many\narrative{not quite sure what these two %mean}).
Successful deployments, including our own in Scotland,
have two attributes in common:
\begin{inparaenum}[(i)]
  \item Networks designs are bespoke, suggesting
    there is no one-size-fits-all solution;
  \item crucially, communities must be invested and
    involved~\cite{Wallace:2015a,Wallace:2015b}.
\end{inparaenum}

Though remote access network research is far from complete, the next question is
increasingly clear: What options do remote, isolated networks have for
`backhaul' to interconnect with the rest of the Internet? We define ``remote''
as far from urban areas where commodified network infrastructure is available.
For example long-distance circuits, if and where they are available, are both
expensive and difficult to reach. Access networks in remote places serve
populations that are dispersed. The lower population density reduces the size of
their user-base when compared to their urban cousins. With no options for
interconnecting with nearby networks to generate economies of scale,
\emph{high-quality} backhaul is prohibitively expensive, if it exists at all.

% , almost by definition, and even in the best case such a network will usually
% not have enough of with a userbase  to the nearest city several hundred
% kilometers away\footnote{For example, the main case-study of this paper, a
% network on the West Coast of Scotland is 240km from the nearest major city that
% has datacentres and any diversity of network infrastructure to speak of.}.

The absence of resource pooling options for remote networks is the focus of this
paper. One such example is operated by the Guifi Foundation~\cite{guifi}. Guifi
operates a regional backbone network as a commons. The abstraction that is
presented to clients is an exchange point implemented over IP. In this type of
network, relationships between end-users are either mediated by Guifi, or
implemented as an overlay.

The \acf{IXP} is a long-standing structure that plays a pivotal role in
facilitating interconnections between networks~\cite{Ager:2012}. We are
motivated by \acp{IXP} for two reasons. First, the primary role of an \ac{IXP}
is economic. Member networks can connect $n$ networks at an IXP with $n$
circuits, rather than arranging for $\frac{n^2}{2}$ circuits independently.
Second, the \ac{IXP} model of multilateral public peering leads to high density
interconnections, and traffic across the exchange that can be comparable in
magnitude to the largest global service providers~\cite{Ager:2012}. Together,
they are an indication that such a topology might be used to improve
inter-connectivity between networks in under-serviced regions, and to pool
otherwise expensive backhaul resources.

In this paper we present RemIX, a \emph{distributed} Internet Exchange for
Remote and rural networks. The RemIX architecture is agnostic to underlying
technologies, embedding the same principles as the successful remote networks it
is designed to serve. It distinguishes itself from \acp{IXP} by the vast
distances permitted between points of presence, and the lower density of member
networks that connect to them. The trade-off between distance and density gives
rise to the idea of \emph{lightweight} points of presence. The lightweight
nature is advantageous, in that as few as two member networks are sufficient to
establish a point of presence. The RemIX switching fabric is agnostic to
physical layer technology and topology.

We describe our RemIX implementation in Scotland. In its current form our
deployment services a \~2000km$^2$ region that spans sea and mountainous
mainland. Implementation details are provided, with motivating rationale, so
that others may benefit from our efforts. Functionally, our implementation
appears to its members as a large Ethernet switch. Crucially, RemIX allows
member networks to establish unmediated relationships between themselves.

In the following sections we further motivate \acp{IXP} as an ideal model. We
then discuss the RemIX architecture in detail. Our deployment is then described,
along with lessons learned. Finally, a broader context of the local environment
is presented before concluding remarks.

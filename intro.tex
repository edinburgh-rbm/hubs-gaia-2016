In remote and rural regions the last-mile problem has been the subject of much
focus. Worldwide, deployments have shown that it is possible to build quality
access networks in the remotest regions~\cite{xxx}. Their technological
foundations range in medium (eg. wired vs wireless), energy (eg. solar,
mainline), and topology (eg. multi-hop, one-to-many). Successful deployments
have two attributes in common: (i) Networks designs are bespoke, suggesting
there is no one-size-fits-all solution; (ii) critically, communites must be
invested and involved~\cite{Wallace:2015a, Wallace2015b}.

Much work remains to be done, though the next question is increasingly clear:
What options does a remote access network have for backhaul, i.e. the
next-to-last mile, to the Internet? 

Backhaul availability is highly constrained in remote regions.
\narrative{economics, expertise; then list projects by facebook and google.}

In Scotland, we have taken inspiration from Internet Exchange Points. In urban
regions, IXPs are used. Describe IXPs, and that they are not-for-profit in most
parts of the world...

In this paper we present 
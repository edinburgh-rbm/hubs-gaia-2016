In remote and rural regions the last-mile problem has been the subject of much
focus. Worldwide, deployments have shown that it is possible to build quality
access networks in the remotest regions~\cite{xxx}. Their technological
foundations range in medium (eg. copper or fibre-optic cabling, licensed
or unlicensed wireless), energy (eg. solar or wind generation or
mains supplied), and topology (eg. multi-hop,
one-to-many\narrative{not quite sure what these two mean}). Successful deployments
have two attributes in common: (i) networks designs are bespoke, suggesting
there is no one-size-fits-all solution; (ii) communites must be
invested and involved~\cite{Wallace:2015a, Wallace2015b}. This last is critical.

Much work remains to be done on the operation of such networks, but
the next question is increasingly clear: What options does a remote,
isolated network have to interconnect with the rest of the Internet?
``Remote'' means far from urban areas where commodified network
infrastructure is available and long-distance circuits, where they are
available, are very expensive. Access networks in remote places serve
a population that is dispersed, almost by definition, and even in the
best case such a network will usually not have enough of a userbase on
its own to pay for a big fat pipe to the nearest city several hundred
kilometers away\footnote{For exampe, the main case-study of this
paper, a network on the West Coast of Scotland is 240km from the
nearest major city that has datacentres and any diversity of network
infrastructure to speak of.}.

\narrative{economics, expertise; then list projects by facebook and
google. why? what has google or facebook done? if we list them then we
have to talk about their economic model which is analogous to the
contribution of the slave trade to the economy of the american south
in the early 19th century}

In the development of the Internet, there is a structure that has
played a pivotal role in facilitating the interconnection of networks
--- the \acf{IXP}. The role of an \ac{IXP} is primarily economic: if
you have $n$ networks that should be connected together, that is
$\frac{n^2}{2}$ circuits that have to be organised between them,
possibly across many sites. Instead, if these networks agree to meet
at a single place, only $n$ such circuits need to be organised
provided that the central place has some sort of neutral, automatic
multipoint to multipoint switching arrangement. This arrangement is
called an \ac{IXP} and any network present there is free to make
arrangements with any other\footnote{One of the authors was a
participant in early \acp{IXP} that were quite literally just an
ethernet switch in a convenient place and any network could connect
under their own steam was welcome for a peppercorn yearly fee.}.

It is unusual today for an \acp{IXP} to be operated by for-profit
entities and though this was not always the
case~\cite{Ager:2012:ALE:2342356.2342393,Chatzis:2013:MIM:2541468.2541473,hayes1997computing}, 
in their modern form they operate as cost centres whose financing is
the collective concern of their members.

In Scotland, we have taken inspiration from Internet Exchange
Points. In urban regions, IXPs are used. Describe IXPs, and that they
are not-for-profit in most parts of the world...

In this paper we present 

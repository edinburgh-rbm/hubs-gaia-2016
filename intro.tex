In remote and rural regions the last-mile problem has been the subject of much
focus. Worldwide, deployments have shown that it is possible to build quality
access networks in the remotest regions~\cite{xxx}. Their technological
foundations range in medium (eg. copper or fibre-optic cabling, licensed
or unlicensed wireless), energy (eg. solar or wind generation or
mains supplied), and topology (eg. multi-hop,
one-to-many\narrative{not quite sure what these two mean}). Successful deployments, including our own in Scotland,
have two attributes in common:
\begin{inparaenum}[(i)]
  \item Networks designs are bespoke, suggesting
    there is no one-size-fits-all solution;
  \item curcially, communities must be invested and
  involved~\cite{Wallace:2015a,
    Wallace2015b}.
\end{inparaenum}

Though remote access network research is far from complete, the next question
is increasingly clear: What options does a remote, isolated network have for
`backhaul' to interconnect with the rest of the Internet? ``Remote'' means far
from urban areas where commodified network infrastructure is available and
long-distance circuits, where they are available, are very expensive. Access
networks in remote places serve a population that is dispersed. The lower
population density reduces the size of their user-base when compared to their
urban cousins. This makes \emph{high-quality} backhaul connectivity
prohibitively expensive, if it exists at all. with no options for
interconnecting with nearby networks to generate economies of scale.

% , almost by definition, and even in the best case such a network will usually
% not have enough of with a userbase  to the nearest city several hundred
% kilometers away\footnote{For example, the main case-study of this paper, a
% network on the West Coast of Scotland is 240km from the nearest major city that
% has datacentres and any diversity of network infrastructure to speak of.}.

The absence of resource pooling options for remote networks is the focus of this
paper. The \acf{IXP} is a structure that plays a pivotal role in facilitating
interconnection between networks~\cite{Ager::2012}. We are motivated by
\acp{IXP} for two reasons. First, the primary role of an \ac{IXP} is economic.
Member networks can connect $n$ networks at an IXP with $n$ circuits, rather
than arranging for $\frac{n^2}{2}$ circuits independently. Second, the \ac{IXP}
model of multilateral public peering leads to high density interconnections, and
traffic across the exchange that can be comparable in magnitude to the largest
global service providers~\cite{Ager:2012}. Together, they are an indication that
such a topology might be used to improve inter-connectivity between networks in
under-serviced regions, and to pool otherwise expensive backhaul resources.

In this paper we describe RemIX, a \emph{distributed} Internet Exchange for
Remote access networks. The RemIX architecture distinguishes itself from \acp{IXP}
 by vast distances between points of presence, and the lower density
of member networks that connect to them. RemIX also embeds the same
principles as the successful remote networks it is designed to serve.
At physical layers the RemIX switching fabric is agnostic to technology and
topology, which allows for bespoke networks. Crucially, at layer-2 RemIX
establishes a relationship with and between its member networks.

In the following we describe the environment in which the  RemIX architecture operates, the details of RemIX and   WHIX, the first deployment of RemIX  in Scotland.

In remote and rural regions the last-mile problem has been the subject of much
focus. Worldwide, deployments have shown that it is possible to build quality
access networks in the remotest regions~\cite{xxx}. Their technological
foundations range in medium (eg. copper or fibre-optic cabling, licensed
or unlicensed wireless), energy (eg. solar or wind generation or
mains supplied), and topology (eg. multi-hop,
one-to-many\narrative{not quite sure what these two mean}). Successful deployments
have two attributes in common:
\begin{inparaenum}[(i)]
  \item networks designs are bespoke, suggesting
    there is no one-size-fits-all solution;
  \item communites must be invested and involved~\cite{Wallace:2015a,
    Wallace2015b}.
\end{inparaenum}
This last is critical.

Much work remains to be done on the operation of such networks, but
the next question is increasingly clear: What options does a remote,
isolated network have to interconnect with the rest of the Internet?
``Remote'' means far from urban areas where commodified network
infrastructure is available and long-distance circuits, where they are
available, are very expensive. Access networks in remote places serve
a population that is dispersed, almost by definition, and even in the
best case such a network will usually not have enough of a userbase on
its own to pay for a big fat pipe to the nearest city several hundred
kilometers away\footnote{For exampe, the main case-study of this
paper, a network on the West Coast of Scotland is 240km from the
nearest major city that has datacentres and any diversity of network
infrastructure to speak of.}.

\narrative{economics, expertise; then list projects by facebook and
google. why? what has google or facebook done? if we list them then we
have to talk about their economic model which is analogous to the
contribution of the slave trade to the economy of the american south
in the early 19th century}

In the development of the Internet, there is a structure that has
played a pivotal role in facilitating the interconnection of networks
--- the \acf{IXP}. As part of the decommissioning of the \ac{NSFNET}
in the early-mid 1990s, four \acp{NAP} were created, each operated by
a large American telephone company (MCI, Sprint, PacBell, Ameritech)
to help ensure that the fledgeling commercial Internet did not suffer
a
partition~\cite{hayes1997computing,Ager:2012:ALE:2342356.2342393}. Being
operated by vested interests, the barriers to participation at
the \acp{NAP} were high in terms of cost and equipment. As an
alternative, \acp{IXP} began to appear in carrier-neutral
facilities. The networking community recognised that mutual
interconnections were desireable and that the function of
the \acp{NAP} was a necessary one, but that they ought not to be
operated by carriers because of the conflicts of interest that
inevitably arose\footnote{One of the authors was a participant in
early \acp{IXP} that were quite literally just an ethernet switch in a
convenient place and any network could connect under their own steam
was welcome for a peppercorn yearly fee.}. There are now several
hundred such \acp{IXP} around the world.

The role of an \ac{IXP} is primarily economic: if
you have $n$ networks that should be connected together, that is
$\frac{n^2}{2}$ circuits that have to be organised between them,
possibly across many sites. Instead, if these networks agree to meet
at a single place, only $n$ such circuits need to be organised
provided that the central place has some sort of neutral, automatic
multipoint to multipoint switching arrangement. This arrangement is
called an \ac{IXP} and any network present there is free to make
arrangements with any other. To avoid the kinds of conflicts of
interest mentioned above, \acp{IXP} are typically organised as a not
for profit entity, and treated as a cost-centre by its members.

This model of multilateral public peering leads to a very high
density of interconnection and traffic flux across the exchange that
in some cases is comparable in magnitude to the largest global
service providers~\cite{Ager:2012:ALE:2342356.2342393}. This
observation is the first indication that such a topology might be
useful for bringing together several remote networks in order that
this increased density of interconnection can be used to pool traffic
to make collective use of expensive resources such as long-distance
circuits. There are some important differences between the environment
of typical \acp{IXP} and that on the West Coast of Scotland:
\begin{inparaenum}[(i)]
  \item there are no data centres there, carrier-neutral or otherwise;
  \item due to geography there is no single place where all of the
    networks could meet.
\end{inparaenum}

In this paper we present 

\todo[inline]{What do we present???}

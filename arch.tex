In this section we present the RemIX architecture. It is instructive
to compare with established IXP architectures, and relate those
benefits in the context of remote access networks.

\subsection{Design Requirements}

Our requirements are shaped by three broad goals:
\begin{inparaenum}[(i)]
  \item establish high-quality backhaul to remote and difficult to
    reach regions;
  \item ensure backhaul affordability for small access networks;
  \item enable the networks to maintain the autonomy that is
    fundamental to their sustainability.
\end{inparaenum}
%Concretely this means that the
Member networks must be able to connect to one or more transit providers.
Members must also be free to arrange and articulate policies among themselves.
These requirements implies that a \emph{logical} concentration of internetwork
connections is desireable, which suggests a shared switching fabric below the
network layer.

\begin{figure*}
  \subfloat[Traditional IXP]{
    \resizebox{0.6\columnwidth}{!}{
      \begin{tikzpicture}
        \ixboxesA
      \end{tikzpicture}
      \label{fig:ixbA}
    }
  } \hfill
  \subfloat[Modern urban IXP]{
    \resizebox{0.6\columnwidth}{!}{
      \begin{tikzpicture}
        \ixboxesB
      \end{tikzpicture}
      \label{fig:ixbB}
    }
  } \hfill
  \subfloat[RemIX]{
    \resizebox{0.6\columnwidth}{!}{
      \begin{tikzpicture}
        \ixboxesC
      \end{tikzpicture}
      \label{fig:ixbC}
    }
  }
  \caption{Comparison of exchange point models. Notice density.}
  \label{fig:ixb}
\end{figure*}

Networks that are capable of connecting to the same location can do so with an
Ethernet switch. This is the basis for traditional \ac{IXP} design (Fig.~\ref{fig:ixbA}) where member networks connect to a central fabric
with their own router that sits inside the IXP facility. Our remote networks
have no such luxury. In response, we take and distribute the contemporary design
of a multi-site \ac{IXP} (Fig.~\ref{fig:ixbB}). A multi-site \acp{IXP}
presents a single logical fabric to its members, implemented with switches
that are joined by private circuits.
% Since we don't have that luxury, we take the contemporary design of a
% multi-site
% \ac{IXP} as in Figure~\ref{fig:ixbB} and take it to a \emph{maximally
% distributed} extreme---Figure~\ref{fig:ixbC}.

The RemIX architecture that emerges (Fig.~\ref{fig:ixbC})
has no large facility nor physical housing. Instead it is
distributed so that \emph{lightweight} points of presence may be established
where there are as few as two members. Members either colocate their border
routers with the exchange switch, or remotely on the far end of a link, as
circumstances dictate.

The set of circumstances motivate the lightweight nature of points of presence.
Since the fabric is distributed, fewer networks that will connect from each
site. High port densities are unnecessary. Simultaneously, space and power are
both at a premium. For example, a remote port into RemIX could be housed in a
small cabinet atop a hill, or in space that is donated by a property owner for
this purpose. Equipment is therefore restricted to the small and
power-efficient.


\subsection{RemIX Components}

\subsubsection{Switching Fabric}

The exchange itself is designed to mimic a distributed Ethernet switch. Multiple
Ethernet-like link options include fibre, 802.11 wireless, licensed wireless,
leased fibre, leased pseudo-wires. The switching fabric may be implemented on
top using BGP-VPLS~\cite{rfc4761} (as we have in
Section~\ref{sec:bgpvpls}) or the
\acs{BATMAN}~\cite{johnson2008simple} or
\acs{TRILL}~\cite{perlman2004rbridges} protocols. The salient feature
between them is MAC address learning to establish an Ethernet switch
similar to the \ac{MEF} E-LAN interface
specification~\cite{mef62,mefes}.

\subsubsection{Member \acp{AS}}

Among traditional \acp{IXP} connected networks are encapsulated into Autonomous
Systems (\acp{AS}). Among RemIX member networks, the policies of the small sized
member networks are
%member networks means that the policies are somewhat
different from the Internet's \ac{DFZ}. In particular, member networks' smaller
routers will be neither be capable of storing the entire Internet routing table,
nor are they likely to announce netblocks large enough to be globally visible.
However, \ac{AS} \emph{encapsulation} enables networks to retain their internal
structures and methodologies, and to safely interconnect with neighbours. Due to
the likelihood of collisions use of private \acp{ASN} is inappropriate for this
purpose~\cite{rfc6996}, as are private IP addresses for the exchange
itself~\cite{rfc1918}.

\subsubsection{Exchange Transit}

RemIX members' IP address spaces will be small, and need some entity to
advertise larger netblocks on their behalf. This suggests a specialized transit
provider to mediate between members and the wider Internet. For this reason
RemIX members form a \narrative{BGP?} confederation with a transit provider that
presents them collectively to upstream providers and other exchange points. This
is unusual for \acp{IXP}: Rarely are transit relationships implemented with
exchange points. However, this is normal in RemIX, and likely necessary to
function in the intended environment. We note that transit service should be
optional to members, with no requirement to purchase said provider's transit as
a condition for joining the exchange. Also, nothing prevents other such
providers from participating.

\subsubsection{Auxilliary Services}

Access networks, upon becoming members and connecting to RemIX,  configure
\ac{BGP} peering amongst themselves. The complexity quickly increases as session
numbers grow with the square of the number of participants. As commonly appears
in \acp{IXP}, a \emph{route-server} repeats announcements from one member to all
others. \ac{BGP} can be a complex, advanced topic in networking. A route
reflector keeps the configuration burden to a minimum. Other useful services
such as \ac{NTP} and looking glasses for assistance in debugging may be offered
in addition.

%\subsection{Structural Benefits}

The overall RemIX architecture is motivated by our own needs in Scotland. In
the next section we present our first-phase implementation of RemIX,
alongside remarks on usability and directions for the future.

%\begin{itemize}
%    \item provides default transit to Internet (useful because of IP
%      limits - have to be careful about if/how to raise IP issue)
%    \item BGP solves $n^2$ connects, gives bilateral arrangements
%    \item (in our case) BGP is also the foundation of exchange fabric, i.e.
%      pt2pt circuits that mimic Ethernet
%\end{itemize}

In this section we present the RemIX architecture. It is instructive to compare with established IXP architectures, and relate those benefits in the context of remote access networks.

\subsection{Design Requirements}

Our requirements are shaped by two broad goals: (i) Establish high-quality
backhaul to remote and difficult to reach regions; (ii) ensure affordability
for the small access networks it is intended to serve. At national and
continental scales, IXPs provide exacty these benefits to their ISP clients. Remix must similarly appear, and behave, in a manner that is identical to an IXP.

%% \begin{figure*}
%% \centering
%%   \subfloat[IXP.]{
%%            \includegraphics[width=\columnwidth]
%%            {figs/logical_layer2_ixp}
%%            \label{subfig:l2ixp}
%%   } \hfill
%%   \subfloat[RemIX.]{
%%            \includegraphics[width=0.9\columnwidth]
%%            {figs/logical_layer2_remix}
%%            \label{subfig:l2remix}
%%   }
%%  \caption{A logical layer-2 view of RemIX as compared to an IXP.}
%% \label{fig:l2}
%% \end{figure*}

At the center of any exchange point is a switch fabric. We set a logical layer-2
view of RemIX depicted in Figure~\ref{subfig:l2remix}, against its IXP
equivalent in Figure~\ref{subfig:l2ixp}. In the IXP, ISPs connect via a switch
architecture that is hosted within a single physical site. ISPs may install
their connected router(s) either on- or off-site.

In contrast, as shown in Figure~\ref{subfig:l2remix}, the switch architecture
itself is distributed over many sites. In addition access network routers sit
outside of the switch architecture. This separation reflects one of the
practical constraints that motivate RemIX. Specifically, that space for
infrastructure in remote regions is absent or problematic. For example, a remote
port into RemIX is more likely to be housed within a small cabinet, or in a
space that is donated by a property owner for this purpose. This subtle separation is critical to the RemIX design. 


\subsection{RemIX Components}

The RemIX architecture is characterized by three main components, beginning with
a public Autonomous System (AS) number to facilitate configuration and ensure
reachability. Subscribers to traditional IXPs are all autonomous systems, with
their own AS numbers. An IXP, being a logical switch, has no need to expose
itself as an exchange point. In contrast, remote subscriber access networks may
be too small to have their own AS numbers, and likely lack the technical
expertise needed to manage a visible AS. 

Physical connections are made Ethernet circuits. Ethernet circuits are provided
by telecommunications providers that may or may not themselves be an ISP. The
circuit characteristics are strongly motivated by the need to generate economies
of scale. For example, a 20Mbps circuit may be prohibitively expensive for a
50-subscriber access network, despite being the minimum sufficient capacity. In
RemIX the remote endpoints of any circuit are resourced to aggregate multiple
access networks. From our own deployment (Section~\ref{sec:impl}) a sample
request-for-quote, with physical characteristics, may be found at~\cite{rfq}.

RemIX then uses the circuits to provide an E-LAN service as defined by the Metro
Ethernet Forum~\cite{mef62}. An E-LAN is a multipoint-to-multipoint Ethernet
virtual circuit. One circuit is sufficient to start, since a single endpoint can
connect with many access network interfaces. The underlying technology used to
enable E-LAN services is left to the implementation. The usability trade-off that emerges is discussed in Section~\ref{subsec:use}.

\subsection{Structural Benefits}

RemIX is agnostic to the underlying technology used to implement E-LAN service.
Our implementation uses VPLS with BGP for signalling and
discovery~\cite{vpls-bgp}. We have found that use of BGP offers a number of
structural benefits. Topologically, BGP eliminates manual configuration of the
n$^2$ connections needed mimic an ethernet exchange fabric.

BGP facilitates auto-discovery, and reduces configuration complexity both within
the exchange fabric, as well as when new subscriber access networks join. This
is a natural consequence of iBGP's need for a full mesh of iBGP
sessions\footnote{Route reflectors and confederations also allow the exchange
fabric to scale.}, within which VPLS configurations can be propogated. In the
presence of a BGP route server, discovery also extends to Internet transit.

The ability to route in BGP based on policy also allows for bilateral
arrangements between access networks. For example, neighbouring access networks
may wish to pool their own network resources, or arrange secondary transit. This
was an unanticipated benefit that, upon reflection, should have been obvious:
Remote access networks are ISPs at small scale. Managed by
autonomous organisations, and driven by the users who subscribe to their
service, they should be able to decide their own policies.

The RemIX architecture was motivated by our own needs in Scotland. In the next section we present our first-phase implementation of RemIX, alongside remarks on usability and directions for the future.

%\begin{itemize}
%    \item provides default transit to Internet (useful because of IP 
%      limits - have to be careful about if/how to raise IP issue)
%    \item BGP solves $n^2$ connects, gives bilateral arrangements
%    \item (in our case) BGP is also the foundation of exchange fabric, i.e.
%      pt2pt circuits that mimic Ethernet
%\end{itemize}

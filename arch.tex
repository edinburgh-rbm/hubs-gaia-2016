In this section we present the RemIX architecture. It is instructive
to compare with established IXP architectures, and relate those
benefits in the context of remote access networks.

\subsection{Design Requirements}

Our requirements are shaped by three broad goals:
\begin{inparaenum}[(i)]
  \item establish high-quality backhaul to remote and difficult to
    reach regions;
  \item ensure affordability for the small access networks it is
    intended to serve;
  \item enable the networks to maintain the autonomy that is
    fundamental to their sustainability.
\end{inparaenum}
Concretely this means that the member networks must be able to
efficiently connect to one or more transit providers and at the same
time continue to make arrangements and articulate policies amongst
themselves. These requirements mean that a \emph{logical}
concentration of internetwork connections is desireable on the one
hand and that there should be a shared switching fabric below the
network layer on the other.

\begin{figure*}
  \subfloat[Traditional IXP]{
    \resizebox{0.6\columnwidth}{!}{
      \begin{tikzpicture}
        \ixboxesA
      \end{tikzpicture}
      \label{fig:ixbA}
    }
  } \hfill
  \subfloat[Modern urban IXP]{
    \resizebox{0.6\columnwidth}{!}{
      \begin{tikzpicture}
        \ixboxesB
      \end{tikzpicture}
      \label{fig:ixbB}
    }
  } \hfill
  \subfloat[RemIX]{
    \resizebox{0.6\columnwidth}{!}{
      \begin{tikzpicture}
        \ixboxesC
      \end{tikzpicture}
      \label{fig:ixbC}
    }
  }
  \caption{Comparison of exchange point models. Notice density.}
  \label{fig:ixb}
\end{figure*}

If all of the networks were capable of connecting to the same
location, the design requirements would be met simply with an Ethernet
switch, not unlike a traditional \ac{IXP} design as in
Figure~\ref{fig:ixbA}. Since we don't have that luxury, we take the
contemporary design of a multi-site \ac{IXP} as in
Figure~\ref{fig:ixbB} and take it to a \emph{maximally distributed}
extreme---Figure~\ref{fig:ixbC}. A RemIX exchange is distributed so
that a lightweight point of presence is established where there are as
little as two members. Members either colocate their border routers
with the exchange edge switch, or remotely on the far end of a radio
link as circumstances dictate.

The RemIX points of presence are \emph{lightweight}. This is
reflective of the practical constraints that motivate it. Because it
must be distributed, at each site there are only a handful of networks
that will connect, so very high port densities are unnecessary. At the
same time space and power are both at a premium. For example, a remote
port into RemIX is more likely to be housed within a small cabinet on
top of a hill, or in a space that is donated by a property owner for
this purpose. Therefore small and power-efficient equipment must be
used.

\subsection{RemIX Components}

\subsubsection{Switching Fabric}
The exchange itself is designed to function like a distributed
Ethernet switch. This is built upon a variety of
Ethernet-like links: community fibre, 802.11 wireless, licensed
wireless, leased fibre, leased pseudo-wires.  There are various
technical alternatives for how this can be implemented such as
BGP-VPLS (see Section~\ref{sec:bgpvpls}) or the \textsc{b.a.t.m.a.n}
protocol but the salient feature is that it functions as a MAC address
learning Ethernet switch, not unlike the \ac{MEF} E-LAN interface
specification~\cite{mef62,mefes}.

\subsubsection{Member \acp{AS}}
As an \ac{IXP}, it can be taken as given that the connected networks
are encapsulated into \acp{AS}. However due to the small size of the
member networks, the policies are somewhat different from in the
Internet's so-called \ac{DFZ}. The networks will typically not have
routers capable of holding the entire Internet routing table, and they
will not typically announce netblocks large enough to be globally
visible. Rather, the purpose is \emph{encapsulation} so that each
network may retain its internal structure and methodologies, and yet
safely interconnect with its neighbours. Due to the
liklihood of collisions it is inappropriate to use private
\acp{ASN}~\cite{rfc6996} for this purpose, nor private IP 
addresses~\cite{rfc1918} for the exchange itself.

\subsubsection{Exchange Transit}
As the members will often not have sufficient amounts of IP address
space to be directly visible on the global Internet, an entity 
can do so must advertise a larger netblock on their behalf. A
specialised transit provider to mediate between the members and the
rest of the Internet is needed. The members form a
\emph{confederation} and the transit provider presents them
collectively to upstream providers and other exchange points. This is
an unusual feature: it is normally the exception rather than the rule
for transit relationships to be implemented over exchange points, but
here it is normal mode of operation. It should be emphasised that this
service is optional and no requirement to purchase transit from this
provider is imposed as a condition for joining the exchange, and
nothing prevents other such providers from participating.

\subsubsection{Auxilliary Services}
When member networks connect to the RemIX, they still have to
configure \ac{BGP} peering amongst themselves. This is a burden
because the number of such sessions grows as the square of the number
of participants. To alleviate this, in common with other \acp{IXP}, a
\emph{route-server} is provided that will repeat announcements
received from one member to all others. Since \ac{BGP} is perceived as
a complex, advanced topic in networking, it is very important to keep
the amount of such configuration to a minimum. Other useful services
such as \ac{NTP} and looking glasses for assistance in debugging may
be made available as well.

\subsection{Structural Benefits}

The features of RemIX described above will be instantly recognisable
to anyone who has participated in a regular \ac{IXP}. This is by
design, as RemIX is an application of a structure that has allowed the
Internet to scale to the rural networking environment. Not only does
encapsulating small community networks in \acp{AS}s mean that they can
present a uniform interface to a transit provider but also that it is
possible for them to cooperate and to share resources and yet for each
to retain the ability to operate their own network as they see
fit. This is very important because keeping as much of operations and
maintenance local is the only way they can be sustainable. In turn
this means that the internal details of each network should suit those
doing the work and what they are comfortable with.

The RemIX architecture was motivated by our own needs in Scotland. In
the next section we present our first-phase implementation of RemIX,
alongside remarks on usability and directions for the future.

%\begin{itemize}
%    \item provides default transit to Internet (useful because of IP 
%      limits - have to be careful about if/how to raise IP issue)
%    \item BGP solves $n^2$ connects, gives bilateral arrangements
%    \item (in our case) BGP is also the foundation of exchange fabric, i.e.
%      pt2pt circuits that mimic Ethernet
%\end{itemize}

%% MF- Moved from 3.3; reads more as final remarks. TBD later.
The features of RemIX described above will be instantly recognisable
to anyone who has participated in a regular \ac{IXP}. This is by
design, as RemIX is an application of a structure that has allowed the
Internet to scale to the rural networking environment. Not only does
encapsulating small community networks in \acp{AS}s mean that they can
present a uniform interface to a transit provider but also that it is
possible for them to cooperate and to share resources and yet for each
to retain the ability to operate their own network as they see
fit. This is very important because keeping as much of operations and
maintenance local is the only way they can be sustainable. In turn
this means that the internal details of each network should suit those
doing the work and what they are comfortable with.

Many working network engineers have had the experience of inheriting a
flat network consisting of a single layer-2 broadcast domain,
invariably plagued by difficult to troubleshoot problems. There is a
superficial similarity to the RemIX architecture, after all it is a
design requirement that members of the \ac{IXP} must be able to
communicate directly without mediation at the IP layer. The similarity
is only superficial, however. Firstly only unicast traffic and
\ac{ARP} is permitted on the exchange \ac{LAN} and this policy,
provided it is enforced, eliminates a large class of potential
problems. Secondly, the requirement that members nominate a specific
\ac{MAC} address for their connection reduces the risk of loops and
broadcast storms. Best practices learned from larger \acp{IXP} such as
quarantine for a new connection until it is shown to be behaving
correctly are valuable for mitigating these risks. The architecture is
such that the extension of the layer-2 broadcast domain extends far
enough to meet the design requirements but not so far as to become
unwieldy.

A high-level criticism that could be directed at the norm of providing
IP transit via the exchange for networks that are not otherwise
visible on the Internet is that it evokes the conflicting interests
that beset the \acp{NAP}. Here too, the similarity is superficial. The
major difference, at least in the Scottish implementation, is that the
transit provider is a cooperative that exists for the benefit of its
members. It does not compete with its members because it does not
provide service to individual end-users, so the economic interests are
aligned and there is no conflict.

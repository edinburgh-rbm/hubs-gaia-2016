%% Relevant HUBS History
In 2012-2013 a collaboration took place between the University of
Stirling and three community networks on the West Coast. They were
physically neighbouring but were not connected to each other; there
was no way for them to share resources or provide mutual operational
support. One had a fast connection via \ac{UHI} but the others could
make no use of it. Though they were each constructed using similar
equipment, they were logically organised very differently: one was
simply a flat layer-2 network, another used static IP routing and the
third used dynamic routing. It was observed~\cite{waites2014ripe} that
to connect them together without disturbing their internal structure
could best be done using \ac{BGP}. And so it was done, to good effect.

At the same time, a similar cluster of community networks was emerging
in the rural areas to the South of Edinburgh. The proximity to a major
urban centre presented an opportunity to create a specialist not for
profit transit provider to cater to their needs. The arrangement up
North with \ac{UHI} was good but somewhat limiting as institutional
networks are seldom designed for transit at their periphery. In order
to have more buying power, the networks wanted to treat with upstream
providers collectively but no commercial providers had suitable
products or offerings. So the networks of the South were organised
together along similar lines as their counterparts in the West
Highlands, but now with a transit provider of their own. This transit
provider was called HUBS and it operates \ac{AS} 60241.

Over the course of time, two of the networks in South Scotland began
to collaborate closely on an operational basis, sharing management and
troubleshooting tasks for each other's equipment. But their only
interconnection was mediated by HUBS, which was a hinderance --- they
did not wish to leak internal details to HUBS but did wish to do so
between themselves. An ad-hoc solution involving hand-crafted circuits
was found, but it was obvious that though the capability to have
bilateral arrangements was very useful, this approach would not scale.
When financing was obtained to build an upgraded backbone to connect
the networks on the West Coast --- who now numbered a dozen --- the
design described herein was developed. The remote networks would be
able to act collectively in the wholesale telecommunications market,
present a uniform interface to their upstream transit provider, and
yet be able to autonomously make arrangements amongst themselves
unmediated at the IP layer.

% In this section we give a brief overview of \acp{IXP}, and then set the context
% for the local Scottish environment before summarizing our early efforts.

%\subsection{Internet Exchange Points (IXPs)}

As part of the decommissioning of the \acs{NSFNET},
%in the early-mid 1990s
four \acp{NAP} were created. They were operated by large
American telephone companies (MCI, Sprint, PacBell, Ameritech) and
designed to prevent partitioning of the commercial
Internet~\cite{Ager:2012,Chatzis:2013}.
The \acp{NAP} were prohibitively expensive and had arbitrary
technological requirements which created barriers to participation.
Soon \acp{IXP} emerged as an alternative. \acp{IXP} appeared in
carrier-neutral facilities allowing dense inter-network connections on
a non-discriminatory basis. Presence at an \acp{IXP} entails freedom
to make bilateral arrangements with any other network also
present. Worldwide, \acp{IXP} now number in the hundreds and are a
fundamental feature in the structure of the Internet.

% Mutual interconnectivity in
% networking community recognised that
% mutual interconnections were desireable and that the function of the \acp{NAP}
% was a necessary one, but that they ought not to be operated by carriers because
% of the conflicts of interest that inevitably arose. There are now several
% hundred such \acp{IXP} around the world.

% The role of an \ac{IXP} is primarily economic: if you have $n$ networks that
% should be connected together, that is $\frac{n^2}{2}$ circuits that have to be
% organised between them, possibly across many sites. Instead, if these networks
% agree to meet at a single place, only $n$ such circuits need to be organised
% provided that the central place has some sort of neutral, automatic multipoint
% to multipoint switching arrangement. This arrangement is called an \ac{IXP} and
% any network present there is free to make arrangements with any other. To avoid
% the kinds of conflicts of interest mentioned above, \acp{IXP} are typically
% organised as a not for profit entity, and treated as a cost-centre by its
% members.

A mirroring of this structure would be useful in joining remote
networks. The increases of interconnection density could then
be used to pool traffic, and make collective use of expensive resources
such as long-distance circuits. However, there are some important
differences between the environment of a typical urban \ac{IXP} and
the rural regions, as in the West Coast of Scotland:
\begin{inparaenum}[(i)]
  \item There are no data centres, carrier-neutral or otherwise;
  \item due to geography there is no single facility where all of the
    networks could meet.
\end{inparaenum}

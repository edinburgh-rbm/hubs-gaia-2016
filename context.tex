In this section we give a brief overview of \acp{IXP}, and then set the context for the local Scottish environment before summarizing our early efforts.

\subsection{Internet Exchange Points (IXPs)}

As part of the decommissioning of the \acs{NSFNET} in the early-mid
1990s, four \acp{NAP} were created. They were operated by large
American telephone companies (MCI, Sprint, PacBell, Ameritech) and
designed to prevent partitioning of the commercial
Internet~\cite{hayes1997computing,Ager:2012}. Participation in
the \acp{NAP} suffered because of high costs and other barriers and
soon \acp{IXP} emerged as an alternative. \acp{IXP} appeared in
carrier-neutral facilities allowing dense inter-network connections on
a non-discriminatory basis. Presence at an \acp{IXP} entails freedom
to make bilateral arrangements with any other network also
present. Worldwide, \acp{IXP} now number in the hundreds and are a
fundamental feature in the structure of the Internet.

% Mutual interconnectivity in
% networking community recognised that
% mutual interconnections were desireable and that the function of the \acp{NAP}
% was a necessary one, but that they ought not to be operated by carriers because
% of the conflicts of interest that inevitably arose. There are now several
% hundred such \acp{IXP} around the world.

% The role of an \ac{IXP} is primarily economic: if you have $n$ networks that
% should be connected together, that is $\frac{n^2}{2}$ circuits that have to be
% organised between them, possibly across many sites. Instead, if these networks
% agree to meet at a single place, only $n$ such circuits need to be organised
% provided that the central place has some sort of neutral, automatic multipoint
% to multipoint switching arrangement. This arrangement is called an \ac{IXP} and
% any network present there is free to make arrangements with any other. To avoid
% the kinds of conflicts of interest mentioned above, \acp{IXP} are typically
% organised as a not for profit entity, and treated as a cost-centre by its
% members.

A mirroring of this structure would be useful in joining remote
networks. The increases of interconnection density could then
be used to pool traffic, and make collective use of expensive resources
such as long-distance circuits. However, there are some important
differences between the environment of a typical urban \ac{IXP} and
the rural regions, as in the West Coast of Scotland:
\begin{inparaenum}[(i)]
  \item There are no data centres, carrier-neutral or otherwise;
  \item due to geography there is no single facility where all of the
    networks could meet.
\end{inparaenum}

\subsection{Motivating Environment}

Scotland comprises 1/3rd of the area of Great Britian, though its population is
less than 10\%. It is also home to 790 islands, 95 of which are inhabited with
\textasciitilde 100,000 people. The Scottish Highlands and Islands, where this
work is currently focussed, consist of mountainous terrain stretching along a
400km north to south corridor. Islands are scattered on the West together with
deep lakes and glens penetrating the mainland to the East.  The economy was
traditionally maritime, and nearly all habitation is at sea level or in the
glens.

Until recently there was little fibre in the region.  Much of the telephone
network in the region was constructed with microwave links. Infrastructure is
improving, though all plans terminate at telephone exchanges. From among those,
fibre-based services are rare. In the medium term future, the only feasible
technology for adequate bandwidth and quality of service is local wireless
distribution.

Starting in 2008, the Tegola project~\cite{tegola} started to
experiment with technology that would enable communities to build
their own wireless networks.
% This included electrical and mechanical infrastructure as well as radio
% equipment.  It rapidly became clear that volunteer communities or small
% businesses could construct and maintain these networks at a small fraction of
% the cost that a centralized organizaton would charge for a number of reasons:
% first, the cost and time of travel to service the relays in remote areas is
% infinitessimal compared with the cost of a helicopter; second site licences and
% wayleaves can usually be negotiated for free by lightweight agreements; third,
The details of the Tegola, and its dissemination to nearby communities, are
omitted due to space constraints. Relevant to this project is the technology
that emerged. Figure~\ref{fig:mhialairigh}, for example, features the type of
robust, inexpensive relay construction that operates in mountainous region, and
that can be constructed by its residents.

\begin{figure}[h]
\centering
 \includegraphics[width=\columnwidth]{figs/mhialairigh-from-behind}
 \caption{A basic relay}
\label{fig:mhialairigh}
\end{figure}

% The ideas were taken up by a number of communities across Scotland
% including those around the Tegola project which extend over 100km of
% the coastline (Fig.~\ref{fig:whixmap}). A typical community would have
% a local distribution network and point-to-point links, often in excess
% of 20km, connected to a set of ADSL lines somewhere near a telephone
% exchange. This was far from ideal, but it was the only affordable
% source of backhaul. Although the communities shared their expertise
% and sometimes their infrastructure, they operated independently.

% Recently, it has become possible to obtain Ethernet services in two
% major towns, but the cost is only reasonable if communities combine
% and buy at wholesale prices.  For this one needs two things: an
% organisational vehicle for the communities act together to achieve
% economies of scale, and a networking infrastructure.

Many communities have since constructed their own local distribution networks
with point-to-point wireless links that can span more than 20km. Expertise is
often shared between them, also infrastructure where feasible, yet they operate
independently. Constrained by availability, they acquire backhaul via ADSL lines
nearby to telephone exchanges. Ethernet services have since emerged in two
larger towns, with wholesale pricing that exceeds the budget of any single
community. A resolution has two components: An organizational vehicle that
combines networks to generate economies of scale, and a supporting network
infrastructure.

% The terrain and the sociology of the Highlands and Island raises some
% important issues for both of these.

We have learned that solutions are complicated by both terrain and by culture.
In particulare we note the following observations.
\begin{itemize}
\item Social aspects and organization of communities can fail to align with the
  design options for generating ``electronic'' or networked communities.
  Physical landscape constrains connectivity, while social and economic groups
  can be determined by vehicles for funding.
  % It is relatively easy to bounce signals
  % back and forth across a loch or glen, but extremely hard to carry
  % them over a 1000m high range of mountains. On the other hand
  % social and economic groupings can be determined by the ways in which
  % they obtain funding.%\footnote{Knoydart is an isolated
    %% peninsula (see Fig.~\ref{fig:whixmap})  of which the North and
    %% West coasts are served by a network based on Loch Hourn and the
    %% South coast is served by Loch Nevis.  Contrast this with the Sleat
    %% peninsula which could, rationally,  be split into three sections,
    %% served by Loch Hourn, Loch Nevis and Eigg, but is run
    %% independently as a single entity with three connections to the
    %% rest of the world.}.
\item Local network infrastructure is non-uniform and varies in complexity.
  % between communities structure their networks in various ways, and the
  % complexity varies from a single hub and spoke toplogy to a network
  % with a dozen links including redundant paths.
  % between relays for redundancy.
% \item Many adjacent communities have successfully duplicated the local
%   access network model. Together they cover a significantly large,
%   contiguous geographical area.
\item Communities that share network resources generally do so in a
  non-systematic or ad-hoc manner.
\end{itemize}

%% Stimulated in part by developments in the Highlands and Islands,
%% communities in the Scottish Borders South of Edinburgh started to
%% develop similar networks.  While the terrain is also mountainous, the
%% communities are within 40km of Edinburgh where fibre services are
%% available.  However these are only affordable if the communities
%% combine to purchase bandwidth at scale. To this end a community
%% interest company, HUBS, was established whose members are the
%% communities itself.  In addition to backhaul, HUBS provides technical
%% help and other services.  It is anticipated that HUBS will extend its
%% reach to serve the West Coast and achieve further economies of scale,
%% for example in the puchase of transit.

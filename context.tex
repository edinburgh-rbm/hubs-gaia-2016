\narrative{Paint a general picture of Scottish landscape, distribution of people/villages, maybe politics(?), etc. May also be appropiate to introduce HUBS here, i.e. sooner rather than later in Sec 4.}

The Scottish Highlands and Islands consist of mountainous  terrain stretching 400km North to South consisting of islands in the West together with deep lochs and glens penetrating the mainland to the East.  The economy was traditionally maritime, and nearly all habitation is at sea level or in the glens.  

Until recently there was little fibre in the region.  Many exchanges were served by microwave links. That has started to change, but the main problem remains that the fibre reaches only to exchanges.  Many communities rely on copper connections to those exchanges in excess of 5km, and there is little prospect of replacing that copper.  In the medium term future local wireless distribution offers the only feasible technology for adequate bandwidth and quality of service~\footnote{There is often no mobile reception, and some sites cannot "see" satellites.}

Starting in 2008, the Tegola project~\cite{tegola} started to experiment with technology that would enable communities to build their own wireless distribution networks.  This included electrical and mechanical infrastructure as well as wireless equipment.  It rapidly became clear that volunteer communities or small businesses could construct and maintain these networks at a small fraction of the cost that a centralized organizaton would charge~\narrative{Do we need to say why?}. The ideas were taken up by a number of communities across Scotland including those around the Tegola project shown in Figure ~\ref{whixmap}, which extend over 100km of the coastline. A typical community would have a local distribution network and point-to-point links, sometimes in excess of 20km, connected to a (possibly bonded) set of ADSL lines somewhere near a telephone exchange. This was far from ideal, but it was the only affordable source of backhaul. Although the communities shared their expertise and sometimes their infrastructure, they operated independently.


Recently, it has become possible to obtain adequate bandwidth at two major exchanges, but the cost is only reasonable if communities combine and buy at "bulk" prices.  For this one needs two things: an organization -- a co-operative -- that will act on behalf of the communities to obtain the economies of scale and a networking infrastructure ...
\narrative{This has already been said in some floating text -- should I incorporate it?}

The terrain and the sociology of the Highlands and Island raises some important issues for both the organizational and networking aspects of HUBS
\begin{itemize} 
\item The social communities do not always align well with the ``electronic'' communities. It is relatively easy to bounce signals back and forth across a loch, but extremely hard to carry them over a 1000m high range of hill.  Knoydart is an isolated peninsula (see Fig.~\ref{whixmap})  of which the North and West coasts are served by a network based on Loch Hourn and the South coast is served by Loch Nevis.  Contrast this with the Sleat peninsula which could, rationally,  be split into three sections, served by Loch Hourn, Loch Nevis and Eigg, but is run independently as a single entity with three connections to the rest of the world.  The reasons for this are social and economic, based on the ways communities can obtain funding.
\item Communities structure their networks in various ways, and the complexity varies from a single point-to multipoint distribution system to a network with six relays  eight or more point-to-point links.  In some cases the networks have more than one physical path between relays fpr redundancy.
\item Communities may temporarily share resources (links or backhaul) when appropriate.
\end{itemize}

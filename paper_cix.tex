\documentclass[10pt]{sig-alternate-05-2015}
%\documentclass{sig-alternate-10pt}
%\documentclass{sigcomm-alternate}

\usepackage[l2tabu,orthodox]{nag}  % Check for use of obsolete LaTeX packages/constructs
\usepackage[utf8x]{inputenc}       % This file is formatted in UTF-8
\usepackage{upquote}               % Fix quotes in verbatim environments

\usepackage{graphicx}              %
\usepackage{amsmath}               % http://ctan.org/pkg/amsmath
%% this breaks tikz... badly...
%\usepackage[all]{onlyamsmath}      % http://ctan.org/tex-archive/macros/latex/contrib/onlyamsmath
\usepackage{url}                   % http://ctan.org/tex-archive/macros/latex/contrib/url
\usepackage[caption=false]{subfig} % http://ctan.org/tex-archive/macros/latex/contrib/subfig
                                   % The subfig package is deprecated in favour of subcaption,
                                   % but as of April 2015 subcaption doesn't work with ACM or
                                   % IEEE style files (this is also the reason for [caption=false])

\usepackage{verbatim, xcolor}
\usepackage{enumitem}
\usepackage{acronyms}
\usepackage{natbib}
\def\bibfont{\small}
\def\bibsep{0pt}
\usepackage{paralist}
\usepackage{whix} %% shiny wcix diagrams
\usepackage{todonotes}
\newcommand{\narrative}[1]{\textcolor{blue}{(narrative: #1)}}

%%MF -- To put all authors on front page, and save space.
\newcommand{\superscript}[1]{\raisebox{1ex}{\ensuremath{{\mathrm{#1}}}}}
%\def\sharedaffiliation{\end{tabular}\newline\begin{tabular}{c}}
\def\we{\superscript{\ast}}
\def\wh{\superscript{\dag}}
\def\wg{\superscript{\ddag}}
\def\ws{\superscript{\natural}}

\begin{document}

% Copyright
\setcopyright{acmcopyright}
%\setcopyright{acmlicensed}
%\setcopyright{rightsretained}
%\setcopyright{usgov}
%\setcopyright{usgovmixed}
%\setcopyright{cagov}
%\setcopyright{cagovmixed}


% DOI
\doi{http://dx.doi.org/10.1145/2940157.2940162}

% ISBN
\isbn{978-1-4503-4423-4/16/08}

% Conference
\conferenceinfo{GAIA,}{August 22-26, 2016, Florianopolis, Brazil}
\CopyrightYear{2016}

\acmPrice{\$15.00}


% --- Author Metadata here ---
% \conferenceinfo{GAIA,}{August 22-26, 2016 Florianopolis, Brazil}
% \CopyrightYear{2016} % Allows default copyright year (20XX) to be over-ridden - IF NEED BE.
% \crdata{978-1-4503-4423-4}  % Allows default copyright data (0-89791-88-6/97/05) to be over-ridden - IF NEED BE.
% --- End of Author Metadata ---

\title{RemIX: A Distributed Internet Exchange\\for Remote and Rural Networks
% \titlenote{(Produces the permission block, and
% copyright information). For use with
% SIG-ALTERNATE.CLS. Supported by ACM.}
}
% \subtitle{(Pronounced 'kicks,' for Community Internet Exchange)
% \titlenote{A full version of this paper is available as
% \textit{Author's Guide to Preparing ACM SIG Proceedings Using
% \LaTeX$2_\epsilon$\ and BibTeX} at
% \texttt{www.acm.org/eaddress.htm}}
%}

%
% You need the command \numberofauthors to handle the 'placement
% and alignment' of the authors beneath the title.
%
% For aesthetic reasons, we recommend 'three authors at a time'
% i.e. three 'name/affiliation blocks' be placed beneath the title.
%
% NOTE: You are NOT restricted in how many 'rows' of
% "name/affiliations" may appear. We just ask that you restrict
% the number of 'columns' to three.
%
% Because of the available 'opening page real-estate'
% we ask you to refrain from putting more than six authors
% (two rows with three columns) beneath the article title.
% More than six makes the first-page appear very cluttered indeed.
%
% Use the \alignauthor commands to handle the names
% and affiliations for an 'aesthetic maximum' of six authors.
% Add names, affiliations, addresses for
% the seventh etc. author(s) as the argument for the
% \additionalauthors command.
% These 'additional authors' will be output/set for you
% without further effort on your part as the last section in
% the body of your article BEFORE References or any Appendices.

\numberofauthors{1} %  in this sample file, there are a *total*
% of EIGHT authors. SIX appear on the 'first-page' (for formatting
% reasons) and the remaining two appear in the \additionalauthors section.
%
\author{
% You can go ahead and credit any number of authors here,
% e.g. one 'row of three' or two rows (consisting of one row of three
% and a second row of one, two or three).
%
% The command \alignauthor (no curly braces needed) should
% precede each author name, affiliation/snail-mail address and
% e-mail address. Additionally, tag each line of
% affiliation/address with \affaddr, and tag the
% e-mail address with \email.
%
% 1st. author
 \alignauthor
  William Waites\we\wh, James Sweet\we\wh, Roger Baig\wg,\\
  Peter Buneman\we, Marwan Fayed\ws, Gordon Hughes\we\\
 \affaddr{{\we}University of Edinburgh, {\wh}HUBS \textsc{c.i.c},
   {\wg}guifi.net, {\ws}University of Stirling}\\
 {\email{\small Corresponding authors: wwaites@tardis.ed.ac.uk, opb@inf.ed.ac.uk, mmf@cs.stir.ac.uk}}
% \email{wwaites@tardis.ed.ac.uk, \{opb,mfourman\}@inf.ed.ac.uk, james@hubs.net.uk, roger.baig@guifi.net, mmf@cs.stir.ac.uk, r.a.simmons@stir.ac.uk}
%  William Waites \\ %\titlenote{Dr.~Trovato insisted his name be first.}\\
% %        \affaddr{School of }\\
% %        \affaddr{1932 Wallamaloo Lane}\\
% %        \affaddr{Wallamaloo, New Zealand}\\
% %        \email{trovato@corporation.com}
% % 2nd. author
%  \alignauthor
%  James Sweet \\
%  \alignauthor
%  Roger Baig \\
%  \and
%  \alignauthor
%  Peter Buneman  %\titlenote{The secretary disavows
% % any knowledge of this author's actions.}\\
% %        \affaddr{Institute for Clarity in Documentation}\\
% %        \affaddr{P.O. Box 1212}\\
% %        \affaddr{Dublin, Ohio 43017-6221}\\
% %        \email{webmaster@marysville-ohio.com}
% % % 3rd. author
%  \alignauthor
%  Marwan Fayed \\ %\titlenote{This author is the
% % one who did all the really hard work.}\\
% %        \affaddr{The Th{\o}rv{\"a}ld Group}\\
% %        \affaddr{1 Th{\o}rv{\"a}ld Circle}\\
% %        \affaddr{Hekla, Iceland}\\
% %        \email{larst@affiliation.org}
% % use '\and' if you need 'another row' of author names
% % % 4th. author
%  \alignauthor
%  Gordon Hughues \\
% %        \affaddr{Brookhaven Laboratories}\\
% %        \affaddr{Brookhaven National Lab}\\
% %        \affaddr{P.O. Box 5000}\\
% %        \email{lleipuner@researchlabs.org}
}
\additionalauthors{Additional authors: Roger Baig (guifi.net), Michael Fourman (University of
Edinburgh), and Richard Simmons (University of Stirling)
% email: {\texttt{jsmith@affiliation.org}}) and Julius P.~Kumquat
% (The Kumquat Consortium, email: {\texttt{jpkumquat@consortium.net}}).
}

\maketitle

\begin{abstract}

The \ac{IXP}, an Ethernet fabric central to the
structure of the global Internet, is largely absent from
% the development of
community-driven collaborative network infrastructure.
%The reasons for this are two-fold.
\acp{IXP} exist in central,
typically urban, environments where strong network infrastructure
ensures high
levels of connectivity.
%Between rural and remote regions,
Between rural and remote networks, separated by distance and terrain,
no such infrastructure exists. In this paper we present RemIX a
distributed \ac{IXP} architecture designed for the community network
environment. We examine this praxis using an implementation in
Scotland, with suggestions for future development and research.

\begin{comment} %MF-comment
The concept of the \ac{IXP} is, so central to the
structure of the global Internet has so far been mostly absent
from the world of community-driven collaborative network
infrastructure development. The dual reasons for this are that
\acp{IXP} typically exist in a central place in an urban area
and community networks are often in rural and remote places,
with long distances, mountains and the sea to contend with.
In this paper we present RemIX a distributed \acp{IXP}
architecture designed for the community network environment
and examine two instances of this praxis in Scotland and
Catalunya.


\par\vspace{\baselineskip}
\todo[inline]{Below deprecated for now in discussion with RB
for purposes of abstract submission deadline.}
Salient differences between HUBS and Guifi:
\begin{itemize}
    \item HUBS lacks a centralising structure, a foundation, with an articulated social contract.
    \item Guifi itself does not provide Internet access
    \item HUBS is an IP Transit provider
    \item If anything, Guifi is more like a (larger version of) WHIX
    \item There is some conflict of interest in HUBS operating WHIX, which Guifi does not suffer from
\end{itemize}
\hrule

Rural access networks are designed to bridge the `last-mile' broadband
gap in regions that are under-served by traditional broadband
providers. Their construction is bespoke, driven by their
beneficiaries, and determined by physical landscape, population
distribution, and monetary budget. Irrespective of their differences,
they are joined by one substantial challenge: connecting to the rest
of the Internet is prohibitively expensive. HUBS \textsc{c.i.c} was
created in Scotland to respond to this. It is a co-operative of access
network members that generates the economies of scale required to
afford backhaul and Internet transit. While intermediation at the IP
layer between the members and the Internet is required for reasons of
scale, it is neither necessary nor desirable amongst the member
networks themselves. In urban areas, networks could interconnect with
each other at an Internet Exchange Point (IXP). In Scotland where the
networks are scattered across 80,000km$^2$ of mountain, field, and sea
it is not so easy. We bridge this gap with a design for a distributed
Internet exchange for access networks in remote places. Doing so
allows for bilateral arrangements for mutual support and assistance
between these networks, and increases the resilience of access network
connectivity. We present the relevant components, and describe our
implementation, so that our efforts may be reproduced.
\end{comment} % MF-comment

\end{abstract}


%
% The code below should be generated by the tool at
% http://dl.acm.org/ccs.cfm
% Please copy and paste CCSXML code instead of the example below.
%
\begin{CCSXML}
<ccs2012>
  <concept>
    <concept_id>10003033.10003034</concept_id>
    <concept_desc>Networks~Network architectures</concept_desc>
    <concept_significance>500</concept_significance>
  </concept>
  <concept>
    <concept_id>10003033.10003106.10003119</concept_id>
    <concept_desc>Networks~Wireless access networks</concept_desc>
    <concept_significance>500</concept_significance>
  </concept>
  % <concept>
  %   <concept_id>10003456.10003462.10003561.10003560</concept_id>
  %   <concept_desc>Social and professional topics~Broadband access</concept_desc>
  %   <concept_significance>500</concept_significance>
  % </concept>
  % <concept>
  %   <concept_id>10003456.10003462.10003561.10003566</concept_id>
  %   <concept_desc>Social and professional topics~Universal access
  %   </concept_desc>
  %   <concept_significance>500</concept_significance>
  % </concept>
</ccs2012>
\end{CCSXML}

\ccsdesc[500]{Networks~Network architectures}
\ccsdesc[500]{Networks~Wireless access networks}
%\ccsdesc[500]{Social and professional topics~Broadband access}
% \ccsdesc[500]{Social and professional topics~Universal access}

\printccsdesc

\keywords{Internet Exchanges (IXP); Community Broadband}

%%%%%%%%%%%%%%%%%%%%%%%%%%%%%%%%%%%

\section{Introduction} \label{sec:intro} In remote and rural regions the last-mile problem has been the subject of much
focus. Worldwide, deployments have shown that it is possible to build quality
access networks in the remotest regions~\cite{xxx}. Their technological
foundations range in medium (eg. copper or fibre-optic cabling, licensed
or unlicensed wireless), energy (eg. solar or wind generation or
mains supplied), and topology.
%(eg. multi-hop, one-to-many\narrative{not quite sure what these two
%mean}).
Successful deployments, including our own in Scotland,
have two attributes in common:
\begin{inparaenum}[(i)]
  \item Networks designs are bespoke, suggesting
    there is no one-size-fits-all solution;
  \item curcially, communities must be invested and
    involved~\cite{Wallace:2015a,Wallace:2015b}.
\end{inparaenum}

Though remote access network research is far from complete, the next question
is increasingly clear: What options does a remote, isolated network have for
`backhaul' to interconnect with the rest of the Internet? ``Remote'' means far
from urban areas where commodified network infrastructure is available and
long-distance circuits, where they are available, are very expensive. Access
networks in remote places serve a population that is dispersed. The lower
population density reduces the size of their user-base when compared to their
urban cousins. With no options for interconnecting with nearby
networks to generate economies of scale, \emph{high-quality} backhaul
is prohibitively expensive, if it exists at all.

% , almost by definition, and even in the best case such a network will usually
% not have enough of with a userbase  to the nearest city several hundred
% kilometers away\footnote{For example, the main case-study of this paper, a
% network on the West Coast of Scotland is 240km from the nearest major city that
% has datacentres and any diversity of network infrastructure to speak of.}.

The absence of resource pooling options for remote networks is the focus of this
paper. The \acf{IXP} is a structure that plays a pivotal role in facilitating
interconnection between networks~\cite{Ager:2012}. We are motivated by
\acp{IXP} for two reasons. First, the primary role of an \ac{IXP} is economic.
Member networks can connect $n$ networks at an IXP with $n$ circuits, rather
than arranging for $\frac{n^2}{2}$ circuits independently. Second, the \ac{IXP}
model of multilateral public peering leads to high density interconnections, and
traffic across the exchange that can be comparable in magnitude to the largest
global service providers~\cite{Ager:2012}. Together, they are an indication that
such a topology might be used to improve inter-connectivity between networks in
under-serviced regions, and to pool otherwise expensive backhaul resources.

In this paper we describe RemIX, a \emph{distributed} Internet Exchange for
Remote access networks. The RemIX architecture distinguishes itself
from \acp{IXP} 
 by vast distances between points of presence, and the lower density
of member networks that connect to them. RemIX also embeds the same
principles as the successful remote networks it is designed to serve.
At physical layers the RemIX switching fabric is agnostic to technology and
topology, which allows for bespoke networks. Crucially, the RemIX
allows member networks to establish unmediated relationships between
themselves.

In the following we describe the environment in which the  RemIX architecture operates, the details of RemIX and   WHIX, the first deployment of RemIX  in Scotland.


\section{\aclp*{IXP}} \label{sec:context} In this section we give a brief overview of \acp{IXP}, and then set the context for the local Scottish environment before summarizing our early efforts.

\subsection{Internet Exchange Points (IXPs)}

As part of the decommissioning of the \acs{NSFNET} in the early-mid
1990s, four \acp{NAP} were created. They were operated by large
American telephone companies (MCI, Sprint, PacBell, Ameritech) and
designed to prevent partitioning of the commercial
Internet~\cite{hayes1997computing,Ager:2012}. Participation in
the \acp{NAP} suffered because of high costs and other barriers and
soon \acp{IXP} emerged as an alternative. \acp{IXP} appeared in
carrier-neutral facilities allowing dense inter-network connections on
a non-discriminatory basis. Presence at an \acp{IXP} entails freedom
to make bilateral arrangements with any other network also
present. Worldwide, \acp{IXP} now number in the hundreds and are a
fundamental feature in the structure of the Internet.

% Mutual interconnectivity in
% networking community recognised that
% mutual interconnections were desireable and that the function of the \acp{NAP}
% was a necessary one, but that they ought not to be operated by carriers because
% of the conflicts of interest that inevitably arose. There are now several
% hundred such \acp{IXP} around the world.

% The role of an \ac{IXP} is primarily economic: if you have $n$ networks that
% should be connected together, that is $\frac{n^2}{2}$ circuits that have to be
% organised between them, possibly across many sites. Instead, if these networks
% agree to meet at a single place, only $n$ such circuits need to be organised
% provided that the central place has some sort of neutral, automatic multipoint
% to multipoint switching arrangement. This arrangement is called an \ac{IXP} and
% any network present there is free to make arrangements with any other. To avoid
% the kinds of conflicts of interest mentioned above, \acp{IXP} are typically
% organised as a not for profit entity, and treated as a cost-centre by its
% members.

A mirroring of this structure would be useful in joining remote
networks. The increases of interconnection density could then
be used to pool traffic, and make collective use of expensive resources
such as long-distance circuits. However, there are some important
differences between the environment of a typical urban \ac{IXP} and
the rural regions, as in the West Coast of Scotland:
\begin{inparaenum}[(i)]
  \item There are no data centres, carrier-neutral or otherwise;
  \item due to geography there is no single facility where all of the
    networks could meet.
\end{inparaenum}

\subsection{Motivating Environment}

Scotland comprises 1/3rd of the area of Great Britian, though its population is
less than 10\%. It is also home to 790 islands, 95 of which are inhabited with
\textasciitilde 100,000 people. The Scottish Highlands and Islands, where this
work is currently focussed, consist of mountainous terrain stretching along a
400km north to south corridor. Islands are scattered on the West together with
deep lakes and glens penetrating the mainland to the East.  The economy was
traditionally maritime, and nearly all habitation is at sea level or in the
glens.

Until recently there was little fibre in the region.  Much of the telephone
network in the region was constructed with microwave links. Infrastructure is
improving, though all plans terminate at telephone exchanges. From among those,
fibre-based services are rare. In the medium term future, the only feasible
technology for adequate bandwidth and quality of service is local wireless
distribution.

Starting in 2008, the Tegola project~\cite{tegola} started to
experiment with technology that would enable communities to build
their own wireless networks.
% This included electrical and mechanical infrastructure as well as radio
% equipment.  It rapidly became clear that volunteer communities or small
% businesses could construct and maintain these networks at a small fraction of
% the cost that a centralized organizaton would charge for a number of reasons:
% first, the cost and time of travel to service the relays in remote areas is
% infinitessimal compared with the cost of a helicopter; second site licences and
% wayleaves can usually be negotiated for free by lightweight agreements; third,
The details of the Tegola, and its dissemination to nearby communities, are
omitted due to space constraints. Relevant to this project is the technology
that emerged. Figure~\ref{fig:mhialairigh}, for example, features the type of
robust, inexpensive relay construction that operates in mountainous region, and
that can be constructed by its residents.

\begin{figure}[h]
\centering
 \includegraphics[width=\columnwidth]{figs/mhialairigh-from-behind}
 \caption{A basic relay}
\label{fig:mhialairigh}
\end{figure}

% The ideas were taken up by a number of communities across Scotland
% including those around the Tegola project which extend over 100km of
% the coastline (Fig.~\ref{fig:whixmap}). A typical community would have
% a local distribution network and point-to-point links, often in excess
% of 20km, connected to a set of ADSL lines somewhere near a telephone
% exchange. This was far from ideal, but it was the only affordable
% source of backhaul. Although the communities shared their expertise
% and sometimes their infrastructure, they operated independently.

% Recently, it has become possible to obtain Ethernet services in two
% major towns, but the cost is only reasonable if communities combine
% and buy at wholesale prices.  For this one needs two things: an
% organisational vehicle for the communities act together to achieve
% economies of scale, and a networking infrastructure.

Many communities have since constructed their own local distribution networks
with point-to-point wireless links that can span more than 20km. Expertise is
often shared between them, also infrastructure where feasible, yet they operate
independently. Constrained by availability, they acquire backhaul via ADSL lines
nearby to telephone exchanges. Ethernet services have since emerged in two
larger towns, with wholesale pricing that exceeds the budget of any single
community. A resolution has two components: An organizational vehicle that
combines networks to generate economies of scale, and a supporting network
infrastructure.

% The terrain and the sociology of the Highlands and Island raises some
% important issues for both of these.

We have learned that solutions are complicated by both terrain and by culture.
In particulare we note the following observations.
\begin{itemize}
\item Social aspects and organization of communities can fail to align with the
  design options for generating ``electronic'' or networked communities.
  Physical landscape constrains connectivity, while social and economic groups
  can be determined by vehicles for funding.
  % It is relatively easy to bounce signals
  % back and forth across a loch or glen, but extremely hard to carry
  % them over a 1000m high range of mountains. On the other hand
  % social and economic groupings can be determined by the ways in which
  % they obtain funding.%\footnote{Knoydart is an isolated
    %% peninsula (see Fig.~\ref{fig:whixmap})  of which the North and
    %% West coasts are served by a network based on Loch Hourn and the
    %% South coast is served by Loch Nevis.  Contrast this with the Sleat
    %% peninsula which could, rationally,  be split into three sections,
    %% served by Loch Hourn, Loch Nevis and Eigg, but is run
    %% independently as a single entity with three connections to the
    %% rest of the world.}.
\item Local network infrastructure is non-uniform and varies in complexity.
  % between communities structure their networks in various ways, and the
  % complexity varies from a single hub and spoke toplogy to a network
  % with a dozen links including redundant paths.
  % between relays for redundancy.
% \item Many adjacent communities have successfully duplicated the local
%   access network model. Together they cover a significantly large,
%   contiguous geographical area.
\item Communities that share network resources generally do so in a
  non-systematic or ad-hoc manner.
\end{itemize}

%% Stimulated in part by developments in the Highlands and Islands,
%% communities in the Scottish Borders South of Edinburgh started to
%% develop similar networks.  While the terrain is also mountainous, the
%% communities are within 40km of Edinburgh where fibre services are
%% available.  However these are only affordable if the communities
%% combine to purchase bandwidth at scale. To this end a community
%% interest company, HUBS, was established whose members are the
%% communities itself.  In addition to backhaul, HUBS provides technical
%% help and other services.  It is anticipated that HUBS will extend its
%% reach to serve the West Coast and achieve further economies of scale,
%% for example in the puchase of transit.


\section{RemIX Architecture} \label{sec:arch} \narrative{Overview of technical details that are generally applicable and transferable.}


\section{RemIX Deployed in Scotland} \label{sec:impl} \narrative{In this section we describe CIX as implemented by HUBS, a non-profit etc, etc... Then (1) Particulars of implementation in Scotland and how they were dealth with (if they exist). (2) What the current network looks like, and anticipated changes and/or {\bf unanticipated stuffs, lessons learned}. (3) Outcomes as they exist so far.}

\subsection{RemIX in Scotland}
\narrative{High level view of WHIX}


\subsection{Usability 'Trade-off'}
\narrative{Our use of cheap proprietary vs open batman-compatible}


\subsection{Use of SDN}
\narrative{describe our use of Ansible; also larger implications for SDN}


\begin{figure}[h]
  \resizebox{\linewidth}{!}{
    \begin{tikzpicture}
      \whixphysicaldiagram
    \end{tikzpicture}
  }
  \caption{
  Diagram of the \acf{WHIX}. Red markers indicate
  the sites at which networks may connect, black lines indicate radio
  links, and orange lines leased 100Mbps or 1Gbps circuits. The areas
  enclosed with dotted lines correspond to the service areas of member
  networks. Note in particular how these service areas correspond more 
  closely to maritime features than landforms.
  }
  \label{whixmap}
\end{figure}

%\section{Guifi in Catalonia} \label{sec:impl} \narrative{In this section we describe Catalonia. OF backbone is the response to the need of a territorial transport}

\begin{figure}[h]
  \resizebox{\linewidth}{!}{
    \begin{tikzpicture}
      \whixphysicaldiagram
    \end{tikzpicture}
  }
  \caption{
  Diagram of the West Highland Internet Exchange. Red markers indicate
  the sites at which networks may connect, black lines indicate radio
  links, and orange lines leased 100Mbps or 1Gbps circuits. The areas
  enclosed with dotted lines correspond to the service areas of member
  networks. Note in particular how these service areas correspond more 
  closely to maritime features than land-forms.
  }
\end{figure}

\section{The Environment}\label{sec:hubs}%% Relevant HUBS History
In 2012-2013 a collaboration took place between the University of
Stirling and three community networks on the West Coast. They were
physically neighbouring but were not connected to each other; there
was no way for them to share resources or provide mutual operational
support. One had a fast connection via \ac{UHI} but the others could
make no use of it. Though they were each constructed using similar
equipment, they were logically organised very differently: one was
simply a flat layer-2 network, another used static IP routing and the
third used dynamic routing. It was observed~\cite{waites2014ripe} that
to connect them together without disturbing their internal structure
could best be done using \ac{BGP}. And so it was done, to good effect.

At the same time, a similar cluster of community networks was emerging
in the rural areas to the South of Edinburgh. The proximity to a major
urban centre presented an opportunity to create a specialist not for
profit transit provider to cater to their needs. The arrangement up
North with \ac{UHI} was good but somewhat limiting as institutional
networks are seldom designed for transit at their periphery. In order
to have more buying power, the networks wanted to treat with upstream
providers collectively but no commercial providers had suitable
products or offerings. So the networks of the South were organised
together along similar lines as their counterparts in the West
Highlands, but now with a transit provider of their own. This transit
provider was called HUBS and it operates \ac{AS} 60241.

Over the course of time, two of the networks in South Scotland began
to collaborate closely on an operational basis, sharing management and
troubleshooting tasks for each other's equipment. But their only
interconnection was mediated by HUBS, which was a hinderance --- they
did not wish to leak internal details to HUBS but did wish to do so
between themselves. An ad-hoc solution involving hand-crafted circuits
was found, but it was obvious that though the capability to have
bilateral arrangements was very useful, this approach would not scale.
When financing was obtained to build an upgraded backbone to connect
the networks on the West Coast --- who now numbered a dozen --- the
design described herein was developed. The remote networks would be
able to act collectively in the wholesale telecommunications market,
present a uniform interface to their upstream transit provider, and
yet be able to autonomously make arrangements amongst themselves
unmediated at the IP layer.


\section{Concluding Remarks} \label{sec:conc} 

%% MF- Moved from 3.3; reads more as final remarks. TBD later.
The features of RemIX described above will be instantly recognizable to anyone
who has participated in a regular \ac{IXP}. This is by design. RemIX is
architected to mirror in under-serviced regions, the benefits of \acp{IXP} in
urban regions. The encapsulation of small community networks in \acp{AS} means
that they can present a uniform interface to a transit provider, cooperate and
share resources. RemIX provides these benefits to members without sacrificing
their independence, a necessary attribute for long-term sustainability.
%
%  and yet for each
% to retain the ability to operate their own network as they see
% fit. This is very important because keeping as much of operations and
% maintenance local is the only way they can be sustainable. In turn
% this means that the internal details of each network should suit those
% doing the work and what they are comfortable with.


\section{Acknowledgements}
This project is supported by the University of Edinburgh, by the
Scottish Government and by local industries including Marine Harvest
Scotland and Benchmark Holdings.  During an incubation period,
\ac{WHIX} is jointly operated by HUBS and the University of Edinburgh.

\section{Additional Authors}
Additional authors: Michael Fourman (University of
Edinburgh), Richard Simmons (University of Stirling).

\bibliographystyle{abbrv}
\bibliography{paper_cix}

\end{document}
